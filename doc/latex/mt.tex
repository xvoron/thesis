\documentclass[12pt,a4paper,twoside,openany]{book}

%%% work %%%

% definice promenne docmode - print or screen mode
\newcommand{\visualmode}[1]{
	\def\docmode{#1}
	} 

% definice promenne langmode - czech or english mode
\newcommand{\thesislanguage}[1]{
	\def\thelanguage{#1} %czech/english
}

% autor prace
\newcommand{\thesisauthor}[1]{
	\def\theauthor{#1}
	\def\theciteauthor{\StrBehind{#1}{ }[\temp]\uppercase\expandafter{\temp}, \StrLeft{#1}{1}.}
}

% vedouci prace
\newcommand{\thesissupervisor}[1]{
	\def\thethesissupervisor{#1}
}

% nazev prace
\newcommand{\thesistitle}[1]{
	\def\thethesistitle{#1}
}


% vlozi titulni list a zadani
\newcommand{\VUTtitle}[2]{
	\pagestyle{empty}
	\includepdf[pages={1}]{#1}
	\ifthenelse{\equal{\docmode}{print}}{\newpage\phantom{blabla}}
	 
	\ifthenelse{\equal{#2}{blank}}
	{
	\newpage\phantom{blabla}
	\newpage\phantom{blabla}
	}{
	\includepdf[pages={1,2}]{#2}}
	} 

% abstrakt + klicova slova + citace
\newcommand{\abstract}[4]{
	\pagestyle{empty}
	\newpage
	\section*{Abstrakt}
	#1
	\section*{Summary}
	#2
	\vspace{20mm}\\
	\section*{Kl\'{i}\v{c}ov\'{a} slova}
	#3
	\section*{Keywords}
	#4
	\vfill
	\ifthenelse{\equal{\thelanguage}{czech}}{
	\section*{Bibliografick\'{a} Citace}
	\theciteauthor \textit{ \thethesistitle}. Brno: Vysok\'{e} u\v{c}en\'{i} technick\'{e} v Brn\v{e}, Fakulta strojn\'{i}ho in\v{z}en\'{y}rstv\'{i}, \the\year. \pageref{LastPage} s., Vedouc\'{i} diplomov\'{e} pr\'{a}ce: \thethesissupervisor.	
	}{
	\section*{Bibliographic citation}
	\theciteauthor \textit{ \thethesistitle}. Brno: Brno University of Technology, Faculty of Mechanical Engineering, \the\year. \pageref{MyLastPage} pages, Master's thesis supervisor: \thethesissupervisor.
	}
    \ifthenelse{\equal{\docmode}{print}}{\newpage\phantom{}} % blank page
	}
	
	
%	BRABLC, M. \textit{Control of Nonlinear Systems using Local Approximation Methods}. Brno, the Czech Republic: Brno University of Technology, Faculty of mechanical engineering, 2016. \pageref{LastPage} pages. Master's thesis, supervisor: doc. Ing. Robert Grepl, PhD..

% prohlaseni + podekovani
\newcommand{\acknowledgements}[2]{
	\pagestyle{empty}
	\newpage\phantom{blabla}
	\vfill
	#1
	\begin{flushright}
		\textbf{\theauthor}\\
		\vspace{1.5cm}
		\large{Brno} . . . . . . . . . . . . . \hfill . . . . . . . . . . . . . . . . .
	\end{flushright}

	\newpage\phantom{blabla}
	\ifthenelse{\equal{\docmode}{print}}{\newpage\phantom{blabla}}\phantom{}
	\vfill
	#2
	\begin{flushright}
		\textbf{\theauthor}
	\end{flushright}
	\ifthenelse{\equal{\docmode}{print}}{\newpage\phantom{blabla}\newpage}
	\newpage
	}

% nastaveni stylu stranky	
\newcommand{\vutpagestyle}{%[1]{
	
	\ifthenelse{\equal{\thelanguage}{czech}}{
		\selectlanguage{czech}
	}{
		\selectlanguage{english}
	}
	
	
	\setcounter{page}{7}
	\pagestyle{plain}
	\renewcommand{\baselinestretch}{1.5}
	\renewcommand{\chaptermark}[1]{\markboth{\MakeUppercase{\thechapter\ ##1}}{}}
	\renewcommand{\sectionmark}[1]{\markright{\MakeUppercase{\thesection\ ##1}}{}}
	\setlength{\abovedisplayskip}{0cm} % skip between equation and text
	\setlength{\belowdisplayskip}{0.5cm} % skip between equation and text
	\setlength{\abovedisplayshortskip}{0cm} % skip between equation and text
	\setlength{\belowdisplayshortskip}{0.5cm} % skip between equation and text
	\setlength{\textfloatsep}{0.5cm} % step between text and figure/table on top of the page
	\setlength{\intextsep}{0.5cm} % step between text and figure/table in text
	\tableofcontents
	\newpage
	\fancyhead{}
	\fancyfoot{}
	\ifthenelse{\equal{\docmode}{print}}
	{
	\fancyhead[LE,RO]{\leftmark}
	\fancyhead[LO,RE]{\rightmark}
	\fancyfoot[RO]{\thepage}
	\fancyfoot[LE]{\thepage}
	}{
	\fancyhead[L]{\leftmark}
	\fancyhead[R]{\rightmark}
	\fancyfoot[C]{\thepage}
	}
}

%\newcommand*{\fullref}[1]{\hyperref[{#1}]{Appendix \nameref*{#1}}} % named refference - no number
%\newcommand*{\fullref}[1]{\hyperref[{#1}]{\autoref*{#1} \nameref*{#1}}} % nambed reference - number

	


\usepackage{subcaption}
\usepackage{graphicx}
\graphicspath{{img/}{../img/}{../../img/}}
\usepackage{caption}
\usepackage{amsmath}

\usepackage{xargs}
\usepackage[pdftex,dvipsnames]{xcolor}  % Coloured text etc.
\usepackage[colorinlistoftodos,prependcaption,textsize=tiny]{todonotes}
%\usepackage{cite}

% Mode:
%\visualmode{print} % print mode 
\visualmode{screen} % screen mode 


% Language:
%\thesislanguage{czech}
\thesislanguage{english}

\thesisauthor{Artyom Voronin}
\thesissupervisor{Martin Brablc}
\thesistitle{Predictive maintenance of pneumatic pistons }


%% included paskages
\usepackage{pdfpages}
\usepackage{xifthen}
\usepackage{fancyhdr}
\usepackage{lastpage} 
\usepackage{titlesec} % allows title formating
\usepackage{lipsum} % some latin text for examples
\usepackage[nottoc]{tocbibind} % add othe chapters (bibliography) to ToC
\usepackage[framed]{mcode} % add formated matlab code
\usepackage{mathtools} % math signs and tools
\usepackage{multicol} % merge columns in table
\usepackage{multirow} % merge rows in table
\usepackage{bm} % bold signs in equations
\usepackage[font=small,skip=0pt]{caption} % smaller distance between caption and figure/table
\usepackage{xstring} % string processing

%% Optional
\usepackage{amssymb} % math signs (check sign)
\usepackage{epstopdf} % allows esp include
\usepackage[shortlabels]{enumitem} % allows itemize indent options
\usepackage{bbding} % special signs (\Checkmark, \XSolidBrush)
%\def\labelitemi{--} % change itmize default marker

\usepackage{setspace}
\usepackage[hang,flushmargin]{footmisc} % footnotes without indentation

%% style settings

% symetrical / asymetrical margins
% a4 size is 210 mm x 297 mm, print size is 170 mm x 250 mm
% \ifthenelse{\equal{\docmode}{print}} 
% {\usepackage[top=2.4cm, bottom=2.3cm, left=1.5cm, right=1.5cm, bindingoffset=10mm]{geometry}}
% {\usepackage[top=2.4cm, bottom=2.3cm, left=2cm, right=2cm, bindingoffset=0mm]{geometry}}

%original VUT print size - 160 mm x 240 mm
\ifthenelse{\equal{\docmode}{print}} 
{\usepackage[top=2.9cm, bottom=2.8cm, left=2cm, right=2cm, bindingoffset=10mm]{geometry}}
{\usepackage[top=2.9cm, bottom=2.8cm, left=2.5cm, right=2.5cm, bindingoffset=0mm]{geometry}}


% title formats
\titleformat{\chapter}[hang]{\Huge\bfseries}{\thechapter}{5mm}{\Huge\bfseries}
\titleformat{\section}[hang]{\Large\bfseries\setstretch{0.6}}{\thesection}{5mm}{\Large\bfseries}
\titleformat{\subsection}[hang]{\large\bfseries\setstretch{0.6}}{\thesubsection}{5mm}{\normalsize\bfseries}

\titlespacing*{\chapter}{0cm}{0cm}{1.5cm} % {command}{left}{before}{after}
\titlespacing*{\section}{0cm}{0.5cm}{0.3cm}
\titlespacing*{\subsection}{0cm}{0.3cm}{0.3cm}

\setlength{\parskip}{0pt} % changes vertical space between paragraphs
\setlength{\headheight}{16pt}

% coloring format
\usepackage[pdftitle={Masters Thesis},
pdfauthor={\theauthor},
%pdftex=true,
%bookmarks=true,a4paper]
linkcolor=black,
colorlinks=true,
breaklinks=true,
urlcolor=black,
citecolor=black,
unicode]%,a4paper]
{hyperref}
%\usepackage[pdftex]{graphicx}
%\DeclareGraphicsExtensions{.png,.pdf,.eps,.bmp,.jpg,.emf}

%% language settings
\ifthenelse{\equal{\thelanguage}{czech}}{
	\usepackage[czech]{babel}
	\usepackage[utf8]{inputenc}
	\usepackage[T1]{fontenc}
	\usepackage{lmodern}


}{
	\usepackage[english,czech]{babel}
}


%% chapter foot/head style
%\ifthenelse{\equal{\docmode}{print}}
%{
%	\fancypagestyle{plain}{%
%		\fancyhead{}
%		\fancyfoot{}
%		\renewcommand{\headrulewidth}{0pt}% Line at the header invisible
%		\renewcommand{\footrulewidth}{0pt}% Line at the footer invisible
%		\fancyfoot[RO]{\thepage}
%		\fancyfoot[LE]{\thepage}
%	}
%}{
%	\fancypagestyle{plain}{%
%		\fancyhead{}
%		\fancyfoot{}
%		\renewcommand{\headrulewidth}{0pt}% Line at the header invisible
%		\renewcommand{\footrulewidth}{0pt}% Line at the footer invisible
%		\fancyfoot[C]{\thepage}
%	}
%}

 % nastaveni stylu dokumentu

\begin{document}
	
\VUTtitle{pages/titulni_color.pdf}{pages/zadani_color.pdf} 
%\VUTtitle{pages/titulni_color.pdf}{blank} % Use when print


\abstract{ % abstrakt cesky

Tato práce se zabývá vytvořením simulačního modelu dvojčinného
pneumatického pístu s mechanickou sestavou, včetně modelů snímačů, s
následujícím odhadem parametrů a aproximací chování demonstračního
zařízení. Dalším cílem je prezentace různých přístupů prediktivní údržby
na datové sadě měřené na demonstračním zařízení. Na
měřený datový soubor se aplikovaly signal-based techniky bez použití simulačního
modelu a model-based metody, které vyžadují použití simulačního
modelu.

Výsledkem této práce je ověření možnosti monitorování stavu zařízení pomocí
nainstalovaných senzorů a vyhodnocení efektivity senzorů z hlediska
přesnosti a finančních nákladů.

	}{ % abstrakt anglicky

This thesis deals with creating a simulation model of a double-acting
pneumatic piston with a mechanical assembly, including the sensors models,
with the following parameter estimation and approximation to the behavior
of a demonstration device. Another goal is the demonstration of various
Predictive Maintenance approaches on a dataset measured on a demonstration
device. Applying signal-based techniques to the measured dataset without using
a simulation model and a model-based method that requires the use of a
simulation model. 

The outcome of this work is the verification of the possibility of
monitoring the device's condition state,  using installed sensors, and
evaluating the efficiency of the sensors in terms of accuracy/cost. 
	}{ % klicova slova česky
dvojčinný pneumatický válec, prediktivní údržba, identifikace a detekce
poruch, zbývající doba použitelnosti, PdM, FDI, RUL
	}{ % klicova slova anglicky
    double-acting pneumatic piston, predictive maintenance, fault detection
    and identification, remaining useful life, PdM, FDI, RUL 
    }{ 
    \subsection*{Uvod}

Od začátku průmyslové revoluce, složitost
výrobní stroje a sériové linky se postupně zvyšovaly a vyžadují
neustálé sledování podmínek systémů z ekonomických důvodů.
Na druhou stranu kritické systémy jako letadla, kosmické lodě,
automobilové systémy, jaderné reaktory a další vyžadují okamžitý poplach
chyba, lokalizovat došlo k chybě a ještě více předvídat možnou budoucnost
poruchy. Tyto požadavky se staly předpoklady pro detekci chyb
a pole Analýza a Prediktivní údržba.

Výrobní proces vždy zahrnoval prvky kontroly chyb a online
monitorování. Od prvních metod detekce poruch, například vizuální
inspekce, dnešní továrny přecházejí na automatizované systémy skládající se z
senzory a výpočetní jednotky k vyhodnocení poruch. Někdy je
kritické pro sledování zpracovatelského zařízení v reálném čase, aby nedošlo k poškození
způsobené chybou nebo anomálií. Každá jednotlivá chyba může způsobit zpomalení
výrobní proces a tím i snížení zisku.

Algoritmy monitorování zařízení v reálném čase vytvořily Fault Detection a
Pole analýzy (FDA). Metody FDA ve většině případů nevyžadují stroj
techniky učení a dokáže detekovat poruchy pomocí základních algoritmů
od Fourierovy analýzy a algoritmů pro kontrolu trendů po složitější
techniky, jako jsou Gaussovské modely směsí.

Vzhledem k množství údajů shromážděných v posledních letech a rozšíření
technologie ukládání dat jako cloudové služby a efektivita výpočtu, to
je možné používat pokročilejší algoritmy pro detekci poruch a
analýza. Pomocí technik klasifikace strojového učení je to možné
izolovat, kde se chyba vyskytuje. Další možnost, která se stane
k dispozici s velkým množstvím dat je odhad zbývajícího užitečného
život (RUL) celého systému. Tyto techniky vedly k predikci
údržba jako snaha o optimální řešení údržby. Aktuální
technický stav zařízení je vždy k dispozici podle informací
extrahované z měřených signálů. Je možné použít aktuální systém
podmínky pro odhad zbývající životnosti v čase nebo vzdálenosti
měření, jako jsou dny, kilometry nebo cykly. Odhadovaný zbytek
životnost dává možnost plánovat údržbu týkající se skutečného systému
podmínky.

Tyto zbývající algoritmy pro odhad životnosti, detekce poruch
metody a techniky modelování a identifikace systémů tvoří nový
pole prediktivní údržby.

Modelování systému umožňuje poskytovat experimenty a vyvíjet řešení
offline před fyzickými implementacemi hardwaru. Nedostupné nebo
náročné implementovat měření lze nahradit generovanými daty
ze simulačního modelu a nakonec pomáhá nasadit robustní algoritmus.

Tato práce poskytuje krátký úvod do detekce poruch a predikce
metodiky údržby a základní terminologie.
Kapitola \ref{ch: teor_surv} popisuje hlavní cíl a problémy
v těchto oblastech a zaměřuje se na podobnosti a rozdíly mezi nimi
dva přístupy.

Vývoj simulačního modelu dvojčinného pneumatického aktuátoru a
porovnání s reálným vybavením pomocí různých přístupů je
popsané v kapitolách 3, 4 a 5.

Následující kapitola 6 ilustruje prediktivní údržbu založenou na signálu
metody využívající různé senzory dostupné v demonstračním zařízení.
Aplikování předzpracování, extrakce funkcí a klasifikační model,
senzory byly hodnoceny z hlediska funkčnosti, přesnosti a ceny.

Techniky prediktivní údržby založené na modelu a simulační model
využití je demonstrováno v kapitole 7. Simulační model je zvyklý
určit zbytkové signály mezi naměřenými daty a simulací
výstup modelu. Pomocí simulačního modelu jsou také údaje o degradaci
generovány a použity při odhadu zbývající životnosti.

\subsection*{Závěr}
Cílem této práce bylo demonstrovat a ověřit detekci poruch a
techniky prediktivní údržby na dvojčinném pneumatickém pístu
montáž jako objekt případové studie.

\subsubsection*{Simulační model}

Jedním z výstupů práce je simulační model
dvojčinný pneumatický pístový systém postavený na základě diferenciálních rovnic
z pneumaticko-mechanické oblasti, modelováno a vyvíjeno pomocí
Software Matlab/Simulink. Simulační model byl odhadnut pomocí
parametry zdravého chování systému. Existuje však možnost
přehodnotit parametry do poruchového stavu a simulovat systém při poruše
stav.

Vzhledem k dostupným naměřeným údajům a výrazně nelineární dynamice
systému, simulační model vykazuje dobrou shodu s naměřeným
data. Na rozdíl od modelu vytvořeného pomocí knihovny Simulink / Simscape je
výrazně méně výpočetně nákladné při zachování číselné hodnoty
stabilita. Tato fakta jsou zásadní, když je odhad parametrů v
pokrok.

Simulační model byl použit k experimentování s chováním systému v systému
různé podmínky, modelovat poruchové situace a generovat data pro návrh
a vyvíjet robustní algoritmy prediktivní údržby.


\subsubsection{Signal-based PdM}
Dalším výstupem je ověření možnosti klasifikace a
detekce poruchového stavu pomocí technik prediktivní údržby,
pomocí metod založených na signálu a modelu.
 
Pokusy byly prováděny na datové sadě měřené na demonstraci
zařízení pomocí sedmi typů senzorů.
  
Metoda založená na signálu je založena na extrakci užitečných informací
přímo ze signálu v časově-frekvenčních doménách. Každý senzor vyžadoval
individuální přístup k předzpracování, extrahování funkcí, hodnocení
vlastnosti a vytváření klasifikačních modelů. Ale obecně existuje
minimální předběžné zpracování potřebné k uchování možných užitečných informací.

Tabulka \ref{sensors_final} obsahuje srovnání čidel ve 2
kategorie, přesnost provedená v datovém souboru testu a náklady na senzory. The
graph \ref{fig:sensors_final_bar} vizualizuje tato data.

Překvapivě všechny senzory vykazovaly přesnost více než 75 \%. Mikrofony
nabízejí vynikající výkon z hlediska nákladů a přesnosti a jsou
vhodné pro instalaci a údržbu.

\begin{figure}[h!]
    \centering
    \includegraphics[width=1\textwidth]{sensors_final_bar.png}
    \caption{Comparison of sensors from accuracy/cost perspective}
    \label{fig:sensors_final_bar}
\end{figure}

\begin{table}[h]
    \centering
    \begin{tabular}{|c|c|c|c|c|c|c|c|}
        \hline
        \textbf{Sensor}   & Acc & Encoder & Flow & Mics & Pressure & Proximity & Strain \\
        \hline
        \textbf{Accuracy [\%]} & 91.6 & 96.1 & 97.2 & 95.8 & 76.6 & 80.5 & 95.0 \\
        \hline
        \textbf{Cost [czk]} & 2x 3500 & 25000 & 6000 & 3x 500 & 1000 & 2x 1000 & 15000 \\
        \hline
    \end{tabular}
    \caption{Comparison of sensors from accuracy/cost perspective}
    \label{tab:sensors_final}
\end{table}

\subsubsection*{PdM podle modelu}

Další částí této práce bylo aplikovat modelové metody a použití a
simulační model pro algoritmy prediktivní údržby. Tyto algoritmy
jsou praktické, když je těžké extrahovat užitečné informace pomocí a
metoda založená na signálu. Nebo je to vhodné v některých případech, pokud tomu rozumíme
dynamiku systému a umím využívat některé systémové proměnné jako
indikátory stavu.

Použití metody extrakce funkcí ve formě nelineární
koeficient identifikačního modelu systému, konkrétně s
Hammerstein-Wiener model, nedal spolehlivé výsledky. Extrahované funkce
nemají statistickou závislost a je nemožné předvídat typ poruchy
použitím této metody na naměřených datech z pneumatického pístu jako případ
studie.

Na druhou stranu zbytkový odhad pomocí simulačního modelu
ukázal vynikající výsledky. Měřený signál polohy byl porovnán s
signál ze simulačního modelu v normálním chování. Tento zbytek
signál byl použit ke klasifikaci poruchového stavu a dosažení 99 \% na a
menší datová sada. Ale vzhledem k výsledkům získaným pomocí signálu
Metoda zbytkového odhadu se může zdát zbytečná. V tomhle
konkrétního případu, z praktického hlediska, zlepšení
výsledek o několik procent nepřináší zásadní změny, ale
doba výpočtu se významně zvyšuje.

Byla také ověřena možnost poruch modelování a simulace senzorů
pomocí simulačního modelu. I když je náročné sbírat chyby
data ze snímače v reálných podmínkách mohou být generována data o poruše
od simulačního modelu a dokonce v kombinaci s primární datovou sadou do
vytvořit syntetický datový soubor.

\subsubsection*{RUL}

Jedním z hlavních cílů prediktivní údržby je odhadnout
zbývající životnost. Původní datová sada neobsahuje záznam o
historická data, která ukazují degradační chování.

Běžným problémem při údržbě pneumatických ovladačů je netěsnost
vzduchu z komory, kde je umístěn píst. Tato situace byla
modelované na simulačním modelu a generovaná data byla použita pro RUL
odhad.

Vygenerovaná datová sada obsahuje 25 simulací s různými poruchami
dynamika. Každá simulace zahrnuje jiný počet cyklů v závislosti na tom
na dynamiku selhání, než dojde k selhání systému. Každý cyklus
obsahuje 10sekundové měření odezvy systému. V
experimentu byl jako předmět zájmu vybrán signál toku. Z
signálu toku, byl vypočítán parametr tvarového faktoru a použit jako a
indikátor stavu.

Výsledkem je, že je možné odhadnout zbývající životnost
generovaný datový soubor degradace pomocí modelu zbytkové podobnosti,
model párové podobnosti a model lineární degradace. Předpověď
výsledky jsou uspokojivé.

\subsubsection*{Další vývoj}

Jako další vývoj by bylo vhodné modelovat odhad
parametry systému po částech ke zlepšení výsledků, s důrazem na
vlastnosti škrticích ventilů a tlumičů s úpravami.

Proveďte měření stavu poruchy úniku vzduchu a sbírejte historické údaje
údaje o degradaci skutečného pneumatického pístu. Následně vyhodnotit
dynamika poruchy způsobené únikem vzduchu. Ověřte možnost
odhad zbývající životnosti pomocí snímače průtoku. Může to být
zajímavá případová studie k ověření možnosti použití odhadu RUL
mikrofony. Pokud je výkon dostupných senzorů nedostatečný,
lze provádět měření tlaku v komoře. Tlak v
komora je přímo závislá na úniku vzduchu z komory, jako
uvedené v rovnici \ref{}. Příklad změn tlaku z
simulační model je znázorněn na obrázku \ref{}.

	} % vyrobi i citaci

\acknowledgements{
    I hereby declare that except where specific reference is made to the work of others,
    the contents of this dissertation are original and have not been submitted in whole or in
    part for consideration for any other degree or qualification in this, or any other university.
    This master’s thesis is my own work and contains nothing which is the outcome of work
    done in collaboration with others.
	}{
	Poděkování..
	} % vlozi prohlaseni a podekovani


\vutpagestyle % prepise styl stranky, vlozi obsah


\chapter{Introduction}
Since the beginning of the industrial revolution, the complexity of
production machines and serial lines has gradually increased and requires
constant monitoring of the conditions of the systems for economic reasons.
On the other hand, critical systems such as aircraft, spacecraft,
automotive systems, nuclear reactors, and others require immediate alarm on
fault, localize occurred fault, and even more predict possible future
faults.  These requirements have become prerequisites for Fault Detection
and Analysis and Predictive Maintenance fields.

The production process always included elements of fault control and online
monitoring. From the first methods of fault detection, such as visual
inspection, today's factories move to automated systems consisting of
sensors and computing units to evaluate the faults. Sometimes it is
critical to monitor processing equipment in real-time to prevent damage
caused by fault or anomaly. Every single fault can cause a slowing down of
the production process and thus reducing the profit.

Device real-time monitoring algorithms have formed the Fault Detection and
Analysis (FDA) field.  FDA methods, in most cases, do not require machine
learning techniques and can detect failures, using fundamental algorithms
from Fourier analysis and trend checking algorithms to more complex
techniques such as Gaussian Mixture Models.

Due to the amount of data collected in recent years and the expansion of
data storage technology as cloud services and computation efficiency, it
has become possible to use more advanced algorithms for fault detection and
analysis. Using classification machine learning techniques, it is possible
to isolate where does the fault occur.  Another option that becomes
available with a large amount of data is to estimate the remaining useful
life (RUL) of the entire system. These techniques have led to predictive
maintenance as an effort for optimal maintenance solutions. The current
technical condition of the equipment is always available by information
extracted from measured signals. It is possible to use current system
conditions to estimate remaining useful life in time or distance
measurements such as days, kilometers, or cycles. Estimated residual
lifetime gives an option to plan maintenance concerning actual system
conditions.

These remaining useful life estimation algorithms, the fault detection
methods and system modeling and identification techniques form a new
predictive maintenance field.

System modeling allows providing experiments and developing solutions
offline before physical hardware implementations. Unavailable or
challenging to implement measurements can be replaced by generated data
from the simulation model and finally helps to deploy a robust algorithm.

This thesis provides a brief introduction to fault detection and predictive
maintenance methodologies and a basic terminology. 
The \ref{ch:teor_surv} chapter describes the main goal and problems
in these areas and focuses on similarities and differences between these
two approaches.

Developing the simulation model of the double-acting pneumatic actuator and
comparing it with the real-life equipment using different approaches is
described in chapter 3, 4, and 5. 

The following chapter 6 illustrates signal-based predictive maintenance
methods using different sensors available in a demonstration device.
Appling preprocessing, feature extraction, and classification model,
sensors were evaluated in terms of functionality, accuracy, and price. 

The model-based predictive maintenance techniques and simulation model
exploitation are demonstrated in chapter 7. The simulation model is used to
determine the residual signals between the measured data and the simulation
model's output. Also, using a simulation model, degradation data are
generated and used in the remaining useful life estimation.

 
%% Artyom Voronin
%%  __     _ _                   
%% / _| __| (_)  _ __  _ __ ___  
%%| |_ / _` | | | '_ \| '_ ` _ \ 
%%|  _| (_| | | | |_) | | | | | |
%%|_|  \__,_|_| | .__/|_| |_| |_|
%%              |_|              
%% Brno, 2021

\documentclass[class=article, crop=false]{standalone}
\usepackage[subpreambles=true]{standalone}
\usepackage{subcaption}

\usepackage{sectsty}
\usepackage{graphicx}
\graphicspath{{img/}{../img/}{../../img/}}
\usepackage{listings}
\lstset{language=Matlab}
\usepackage{hyperref}
\usepackage{amsmath}
\usepackage{import}
\usepackage{subfiles}
\usepackage[utf8]{inputenc}
\usepackage[english]{babel}

%\usepackage[]{todonotes}
\usepackage{xargs}
\usepackage[pdftex,dvipsnames]{xcolor}  % Coloured text etc.
\usepackage[colorinlistoftodos,prependcaption,textsize=tiny]{todonotes}

%\usepackage[square, numbers]{natbib}
%\bibliographystyle{unsrtnat}
%\usepackage[nottoc]{tocbibind}

%\usepackage{biblatex}
%\addbibresource{./citations.bib} %usage \cite{test}
% Margins
% 
\topmargin=-0.45in
\evensidemargin=0in
\oddsidemargin=0in
\textwidth=6.5in
\textheight=9.0in
\headsep=0.25in

%\title{PM and FDI comparison}
%\author{Artyom Voronin} 
%\date{}

\begin{document}
\tableofcontents

% -----------------------------------------------------------------------------
% 
% -----------------------------------------------------------------------------
\section{Theoretical Survey}
Why FDA and PdM are useful? Similarities/Differences?
The relative arrangement PdM and FDI methods representing in following
figure \ref{fig:fdi_pm}
\begin{figure}[h!]
    \centering
    \includegraphics[scale=0.3]{FDI_PM.png}
    \caption{PM and FDI }
    \label{fig:fdi_pm}
\end{figure}



% -----------------------------------------------------------------------------
% 
% -----------------------------------------------------------------------------

\subsection{Problem Definition}

In practice, many types of machinery require some calibration for adequately working,
and online monitoring and classification algorithms can find the problem.

% XXX FaultDetectionMethods-ALiteratureSurvay.pdf


Smart systems = Sensors, but sensors only not doing the system smart. Smart
is combination of sensors, processing signals, extraction useful information and
some decision based on this information.

What kind of sensors to use and which sensor is the best based on
cost/technical efficient perspective.

\begin{itemize}
    \item Fault
    \item Failure   
    \item Malfunction
\end{itemize}

\subsubsection{Types of Faults}
\begin{itemize}
    \item Plant
    \item Sensor
    \item Combination
\end{itemize}

\subsubsection{Types of Maintenance}
\begin{itemize}
    \item{Reactive (fails than fix)}
    \item{Preventative (schedules)}
    \item{Condition-based (based on assess of system)}
    \item{Predictive (based on model that predict failure)}
\end{itemize}






% -----------------------------------------------------------------------------
% 
% -----------------------------------------------------------------------------

\subsection{Fault Detection and Analysis}
Fault Detection and Analysis, FDA. (Fault detection and isolation, FDI) 

\paragraph{FD not new}
FD exists from 60th.

\subsubsection{Goals}

\begin{itemize}
    \item{Fault detection: Detect malfunctions in real time, as soon and as
        surely as possible}
    \item{Fault isolation: Find the root cause, by isolating the system
        components whose operation mode is not nominal}
    \item{Fault identification: Estimation the magnitude (size) and type or
            nature of the fault}
\end{itemize}

%\textbf{Fault \footnotemark detection and isolation} is subfield of control
%engineering, identifying when fault  has occurred, and pinpointing the type
%of fault and its location.

\footnotetext{\textbf{Fault} - not acceptable deviation of at least one characteristic or
parameter of the system from the standard condition.}

\subsubsection{Methods}
Figure \ref{fig:fault_detection} introduce 2 main approaches:

\begin{itemize}
\item{Model-based FDI (compare data with healthy-model)}
\item{Signal processing based FDI (using math methods to extract information
    about the fault from data)}
\end{itemize}

\begin{figure}[h]
    \centering
\begin{subfigure}{0.5\textwidth}
    \includegraphics[width=0.9\linewidth]{model_based.png}
    \caption{Model-based fault detection}
    \label{fig:model_based}
\end{subfigure}%
\begin{subfigure}{0.5\textwidth}
    \includegraphics[width=0.9\linewidth]{signal_based.png}
    \caption{Signal-based fault detection}
    \label{fig:signal_based}
\end{subfigure}
\caption{Fault detection common approaches}
\label{fig:fault_detection}
\end{figure}

\paragraph{Signal-Based methods}
\begin{itemize}
    \item Limit Checking and Trend Checking
    \item Data Analysis (PCA)
    \item Spectral Analysis
    \item Parametric Signal Models
    \item Pattern Recognition (kNN, ANN, SOM)
\end{itemize}

\paragraph{Model-Based methods}
We know system structure. Faults modeled as some system variables changes.
Parameter estimation

\paragraph{Knowledge-Based methods}
We know some Expert Knowledge about system behavior. Fuzzy,
confidence-numbers, probability density functions, logic
fault-symptom-tree, if-then rules.

The result of FDI is the detection and identification of faults that occur
during the operation of the device. Subsequently data is processed using
Fault Tolerance and Predictive maintenance methods.

\textbf{Fault Tolerance}: Provide the system with the hardware architecture and
  software mechanisms which will allow, if possible to achieve a given
  objective not only in normal operation, but also in given fault
  situations.


\subsubsection{Condition Monitoring}
Answer to question:"How does system operate now?"
CM gives Diagnostic methods that provides alarm or warning, but not
prognostic forecast about the future behavior (Not RUL).

But collected Condition Monitoring information can give information about
system degradation.


There is a optimization between technical and financial possibilities in a
specific situation.

FMECA (Failure Mode, Effect and Criticality Analysis) \\
FTA (Fault Tree Analysis) \\
RCA (Root Cause Analysis) \\


% -----------------------------------------------------------------------------
% 
% -----------------------------------------------------------------------------


\subsection{Predictive maintenance}
\textbf{Predictive maintenance (PdM)} is cost-effective maintenance strategy that
predicts time to failure and warns of an anticipated location where this
could occur.

\subsubsection{Goals}
The are two main goals of Predictive maintenance, RUL (remaining useful
life) estimation and identification where the future failure can appear, or what is
the reason of decreasing RUL. 
As a result of PdM is RUL representing of number cycles, days, or some time
period before fault occurred. And probability where this fault can appear.

Predict where, when and what is the reason of failure (identify primary
factors).


\textbf{Predictive maintenance development sequence}:
\begin{enumerate}
    \item{Collect data (using sensors, math model)}
    \item{Process data (clean up data)}
    \item{Identify Condition Indicators CI}
        \begin{itemize}
            \item{Signal-based CI}
            \item{Model-based CI}
        \end{itemize}
    \item{Fit model (ML techniques)}
    \item{Deploy monitoring and integrate}
    \item{Dashboard (UI)}
\end{enumerate}


\subsubsection{Methods}
There are couples of signal processing and analyzing methods that used in
both PdM and FDI. For example:

\paragraph{Signal-Based} approach is suitable in situation when we have
measurements from system in different operating conditions. 
But there is a problem that Signal-Based approach enable to classify and
learn patterns observed in training dataset. 


\paragraph{Model-Based} approach is to use physical failure models. This
models do not require a large dataset of failure data. And they can work in
situations never observed before. 


%\begin{itemize}
%    \item Spectral Analysis
%    \item Wavelet Analysis
%    \item Wavelet transform
%    \item FFT
%    \item Short Term Fourier Transform
%    \item Gabor Expansion
%    \item Wigner-Ville distribution
%    \item Correlation
%    \item High resolution spectral analysis
%    \item Waveform Analysis
%    \item Time-Frequency Analysis
%    \item PCA
%    \item Machine Learning techniques:
%        \begin{itemize}
%            \item kNN
%            \item ANN
%        \end{itemize}
%\end{itemize}

\subsubsection{Condition Indicators}
Features in PdM field are called Condition Indicators or CI.
Condition Indicators are features extracted from the signals, representing some
system behavior and hides some information about system processing.

Condition indicators represented by three main domain. There are Time
domain, Frequency domain, Time-Frequency domain Condition Indicators.

\begin{itemize}
    \item Time-domain
    \item Frequency-domain
    \item Time-frequency
\end{itemize}

\subsubsection{Fault Classification}

\subsubsection{Remaining useful life}
RUL goal is remaining time before machine requires maintenance. Not only
predict but provide a confidence bound.

\paragraph{RUL Models}:
% https://www.mathworks.com/company/newsletters/articles/three-ways-to-estimate-remaining-useful-life-for-predictive-maintenance.html

Inputs are condition indicators and models depends on data: 1. Lifetime,
Run-to-failure, known threshold for CI.

\begin{itemize}
    \item Similarity model 
    \item Survival model
    \item Degradation model
\end{itemize}


\subsection{Digital twin}
% https://explore.mathworks.com/digital-twins-for-predictive-maintenance

Digital twin is digital representation of the real life system. Can be
represented as a component, a system of components, or as a system of
system.  

\paragraph{Updating digital twin with incoming data} 

Digital twin can be updated with incoming data from sensors. Fitting model
to new data, digital twin represents the current condition state of the
real world object.

\todo[inline]{Rewrite}
Digital twin can hold historical data about behavior of a system
and can be used for simulation system operation in different conditions,
for designing control and simulate future behavior. (RUL, "What-if")

Digital Twins are helpful in the field of Anomaly Detection and Predictive
Maintenance.

Mathematical model of the real world system can be created using different
approaches. Modeling based on Physical modeling (Simscape) data-driven
modeling where system is represented as a "Black box" or some combination
of this approaches.
Model with estimated parameters uses for simulation system behavior in
different working conditions and with different faults during working
process.

\subsubsection{Using Digital Twin in PdM}

Measured data, Generated data from mathematical model, or Synthetic data
(Combination of measured and generated) can be used for assessment of
Condition Indicators. 

%\subsubsection{Detect and Diagnose Faults}
%Using condition indicators on Test data we can analyze actual system state.
%Designing algorithm is iterative process when you try different
%combinations of condition indicators and different models to evaluate best
%results.

\subsection{Comparison PdM and FDA approaches}

\subsection{Application field}

\end{document}


\chapter{Demonstration Device Overview}

The case study of this thesis is the double-acting pneumatic piston, with a
pneumatic circuit and mechanical assembly driven by a piston.  Figure
\ref{} is a schematical representation of the system. Figure \ref{} is a 3D
render of the system.


There are seven types of sensors located on the system. Table 1 describes a
sensor purpose, signal name in the datastore, and the signal unit. 

\begin{table}[h]
    \centering
    \begin{tabular}{|c|c|c|c|}
\hline
\textbf{Sensor} & \textbf{Unit} & \textbf{Description} & \textbf{Name} \\
\hline
Encoder       & m     & displacement measurement               & LeverPosition \\
Encoder       & m/s   & velocity calculated from displacement  & LeverVelocity \\
Accelerometer & g     & accelerometer on moving part           & AccelerometerMovin\_axisZ/Y \\ 
Accelerometer & g     & accelerometer on static part           & AccelerometerStatic\_axisZ/Y \\ 
Flow Sensor   & l/min & air flow extrusion to A chamber        & FlowExtrusion \\
Flow Sensor   & l/min & air flow contraction from A chamber    & FlowContraction \\
\hline
    \end{tabular}
    \caption{Sensors overview}
    \label{tab:sensors_tab}
\end{table}


The dataset measured on the system contains almost five thousand thousand
measurements in different operating conditions. Each measurement includes a
10-second recording of moving the pistol up and down. This data was given
in the format of massive files with the ".mat" extension, which was divided
into files contains only one measurement.  The divided dataset is easier to
maintain, and Matlab recommends this type of datastores called Data
Ensemble \ref{}.

The measured examples are shown in figures 2,3, and 4.

% Artyom Voronin
%  __  __           _      _ 
% |  \/  | ___   __| | ___| |
% | |\/| |/ _ \ / _` |/ _ \ |
% | |  | | (_) | (_| |  __/ |
% |_|  |_|\___/ \__,_|\___|_|
%                            
% Brno, 2020

\documentclass[class=article, crop=false]{standalone}
\usepackage[subpreambles=true]{standalone}

\usepackage{subcaption}
\usepackage{caption}
\usepackage{sectsty}
\usepackage{graphicx}
\graphicspath{{img/}{../img/}{../../img/}}
\usepackage{listings}
\lstset{language=Python}
\usepackage{hyperref}
\usepackage{amsmath}
\usepackage{import}
\usepackage{subfiles}
\usepackage[utf8]{inputenc}
\usepackage[english]{babel}

\usepackage{xargs}
\usepackage[pdftex,dvipsnames]{xcolor}  % Coloured text etc.
\usepackage[colorinlistoftodos,prependcaption,textsize=tiny]{todonotes}

%\usepackage[square, numbers]{natbib}
%\bibliographystyle{unsrtnat}
%\usepackage[nottoc]{tocbibind}

%\usepackage{biblatex}
%\addbibresource{citations.bib} %usage \cite{test}
% Margins
% 
\topmargin=-0.45in
\evensidemargin=0in
\oddsidemargin=0in
\textwidth=6.5in
\textheight=9.0in
\headsep=0.25in

%\title{Models}
%\author{Artyom Voronin} 
%\date{}

\begin{document}
\tableofcontents

% -------------------------------------------------------
% Section
% -------------------------------------------------------
\section{First Principle Approach (15 pages)}
\paragraph{First Principles} (White-Box) \\
% TODO Isermann Fault Detection str 72
Simplification, Liniarization, Reduction, Parameter Estimation. \\
SimScape (Physical modeling), Simulink (Differential equations).

\paragraph{Data-Driven modeling}(Black-Box) \\ % In many cases better for prediction
% TODO Isermann Fault Detection str 72
Measurements, Identification.

% Nonlinear dynamic modeling  Isermann FDS str. 84

\subsection{Pneumatic piston system overview}
%\begin{figure}[h!]
%    \centering
%    \includegraphics[width=0.5\textwidth]{model_draw_vec.png}
%    \caption{Schematic model}
%    \label{fig:model_draw}
%\end{figure}







% -------------------------------------------------------
%  __ _  ___ _ __   ___ _ __ __ _| |
% / _` |/ _ \ '_ \ / _ \ '__/ _` | |
%| (_| |  __/ | | |  __/ | | (_| | |
% \__, |\___|_| |_|\___|_|  \__,_|_|
% |___/                             
% -------------------------------------------------------
\subsection{General physical principles}
\paragraph{Equation of state}
Generally $pV=nR_mT$ but for air, using ideal gas constant $R=287.1 [Jkg^{-1}K^{-1}]$
state equation can be rewrite as \ref{eq:equation_of_state}.
\begin{align}
    pV = mRT
    \label{eq:equation_of_state}
\end{align} 

\paragraph{Isothermal process}
For isothermal process \ref{eq:isothermal_process}:
\begin{align}
    p_1 V_1 = p_2 V_2 = const
    \label{eq:isothermal_process}
\end{align}

\paragraph{Adiabatic process}

Adiabatic process \ref{eq:adiobatic_process}:
\begin{align}
     p_1V_1^{\kappa} =  p_2V_2^{\kappa} = const
    \label{eq:adiobatic_process}
\end{align}

where $\kappa = c_p/c_v$ is a heat capacity ratio. Another
important equation is Mayer's relation $c_p = c_v + R$.

\paragraph{Thermodynamics}
%\begin{tabular}{ |c|c|c| }
%    \hline
%    $p$                     & $Pa$              & pressure \\
%    $V$                     & $m^3$             & volume \\
%    $m$                     & $kg$              & mass \\
%    $n$                     & $mol$             & amount of substance \\
%    $R$                     & $Jkg^{-1}K^{-1}$  & ideal gas constant \\
%    $r$                     & $Jkg^{-1}K^{-1}$  & mass-specific gas constant \\
%    $T$                     & $K$               & temperature \\
%    $S$                     & $m$               & area \\
%    $z$                     & $m$               & height \\
%    $w$                     & $ms^{-1}$         & flow speed \\
%    $H$                     & $J$               & enthalpy \\
%    $\nu$                   & $m^3kg^{-1}$      & specific volume \\
%    $Q$                     & $J$               & heat shared with
%                                                    environment \\
%    $W_T$                   & $J$               & work \\
%    $c_p$                   & $Jkg^{-1}K^{-1}$  & is the specific heat
%                                                    at constant pressure \\
%    $c_v$                   & $Jkg^{-1}K^{-1}$  & is the specific heat at constant volume\\
%    $g=9.81$                & $ms^{-2}$         & gravity acceleration \\
%    $\kappa=1.4\text{(air)}$& $-$               & heat capacity ratio
%                                                    (isentropic expansion factor)\\
%    \hline
%\end{tabular}


\paragraph{Bernoulli's principle}
Bernoulli's principle \ref{eq:bernoullis_principle}:
\begin{align}
    H_1 + \frac{mw_1^2}{2} + mgz_1 + Q = H_2 + \frac{mw_2^2}{2} + mgz_w +
    W_T
    \label{eq:bernoullis_principle}
\end{align}

\begin{align}
    H_1- H_2 = -\int_1^2 V dp = c_p(T_1-T_2) = c_p T_1(1-\frac{T_2}{T_1})
    \label{eq:etalpi_sub}
\end{align}

Differential form:
\begin{align}
    \nu dp + w dw + g dz + dw_T = 0
\end{align}


\paragraph{Fluid mechanics}
%\begin{tabular}{ |c|c|c| }
%    \hline
%    $\dot{m}$                   & $kgs^{-1}$  & mass flow \\
%    $c$                         & $ms^{-2}$   & speed of sound \\
%    $w_k$                       & $ms^{-2}$   & critical flow velocity \\
%    $\psi$                      & $-$         & flow coefficient \\
%    $\psi_{max}$                & $-$         & critical flow coefficient \\
%    $\beta$                     & $-$         & ration of pressure
%                                                    differential \\
%    $\beta_k$                   & $-$         & critical ratio of pressure
%                                                    differential \\
%
%    \hline
%\end{tabular}

Continuity equation \ref{eq:continuity_equation}: 
\begin{align}
    \dot{m} = S_1 w_1 \rho_1 = S_2 w_2 \rho_2 = const
    \label{eq:continuity_equation}
\end{align}

\paragraph{Air expansion from tank}
Assuming $W_T = 0, z_1 = z_2, Q = 0$ conditions and combine with
\ref{eq:bernoullis_principle} we will get \ref{eq:w2} equation:

\begin{align}
    w_2 = \sqrt{2(H_1 - H_2)}
    \label{eq:w2}
\end{align}

\begin{align}
    w_2 =
    \sqrt{2RT_1(\frac{\kappa}{\kappa-1})(1-(\frac{p_2}{p_1})^\frac{\kappa-1}{\kappa})}
    \label{eq:w2_final}
\end{align}

\begin{align}
    \rho_2 = \frac{p_1}{RT_1} (\frac{p_2}{p_1})^{\frac{1}{\kappa}}
    \label{eq:rho2}
\end{align}

Together \ref{eq:continuity_equation} \ref{eq:w2_final} \ref{eq:rho2}:
\begin{align}
    \dot{m} = S p_1 \sqrt{\frac{2}{RT_1}} \cdot
    \sqrt{\frac{\kappa}{\kappa-1}\left(\left(\frac{p_2}{p_1}\right)^\frac{2}{\kappa} -
    \left(\frac{p_2}{p_1}\right)^\frac{\kappa + 1}{\kappa}\right)}
    \label{}
\end{align}

where: 
\begin{align}
    \psi\left(\frac{p_2}{p_1}\right) =  
    \sqrt{\frac{\kappa}{\kappa-1}\left(\left(\frac{p_2}{p_1}\right)^\frac{2}{\kappa} -
    \left(\frac{p_2}{p_1}\right)^\frac{\kappa + 1}{\kappa}\right)}
    \label{eq:psi}
\end{align}

Finally \ref{eq:mass_flow}:
\begin{align}
    \dot{m} = Sp_1\sqrt{\frac{2}{RT_1}} \psi\left(\frac{p_2}{p_1}\right)
    \label{eq:mass_flow}
\end{align}

\paragraph{Critical flow velocity}
Speed of sound:
\begin{align}
    c = \sqrt{\frac{dp}{d\rho}} = 
    \sqrt{\frac{\kappa p}{\rho}} = \sqrt{\kappa R T}
    \label{eq:speed_of_sound}
\end{align}

Assume $c=w_2$ (\ref{eq:w2_final}, \ref{eq:speed_of_sound}) we will get the
critical flow velocity:
\begin{align}
    &c_2 = w_k = \sqrt{\kappa RT} =
    \sqrt{2RT_1\frac{\kappa}{\kappa-1}-2w_k^2\frac{1}{\kappa-1}} \\
    &w_k^2 = 2RT_1\frac{\kappa}{\kappa-1}-2w_k^2\frac{1}{\kappa-1} \\
    &w_k = \sqrt{2RT_1\frac{\kappa}{\kappa-1}} = \sqrt{2p_1 \nu_1 \frac{\kappa}{\kappa + 1}}
    \label{eq:wk}
\end{align}


For calculating critical pressure ratio assume $w_k = w_2$ \ref{eq:wk}
\ref{eq:w2_final}:
\begin{align}
    &\sqrt{2RT_1\frac{\kappa}{\kappa-1}}  = 
    \sqrt{2RT_1 \frac{\kappa}{\kappa-1}
    \left(1-\left(\frac{p_2}{p1}\right)^{\frac{\kappa+1}{\kappa}}\right)} \\
    &\left(\frac{p_2}{p1}\right)^\frac{\kappa-1}{\kappa} = \frac{2}{\kappa+1} \\
\end{align}
\begin{align}
    &\left(\frac{p_2}{p1}\right)_k =
    \left(\frac{p_k}{p1}\right) =
    \left(\frac{2}{\kappa+1}\right)^\frac{\kappa}{\kappa-1}=\beta_k
    \label{eq:beta_k}
\end{align}

Critical pressure condition is $p_k = p_1 \beta_k$.

Applying \ref{eq:beta_k} to \ref{eq:psi}:
\begin{align}
    &\psi_{max} (\beta_k) = 
    \left(\frac{2}{\kappa+1}\right)^\frac{\kappa}{\kappa-1}\sqrt{\frac{\kappa}{\kappa+1}}
\end{align}

For air $\beta_k = 0.528, \psi_{max} = 0.484$


Final equation for $\psi$:
\begin{align}
    \psi\left(\frac{p_2}{p_1}\right) = 
    \begin{cases}
    \sqrt{\frac{\kappa}{\kappa-1}\left(\left(\frac{p_2}{p_1}\right)^\frac{2}{\kappa} -
    \left(\frac{p_2}{p_1}\right)^\frac{\kappa + 1}{\kappa}\right)} & 0.528
    <\frac{p_2}{p_1} \le 1 \\
    \left(\frac{2}{\kappa +1}\right)^{\frac{1}{\kappa+1}}
    \sqrt{\frac{\kappa}{\kappa +1}} & 0 \ge \frac{p2}{p1} \le 0.528\\
    \end{cases}
\end{align}


% -------------------------------------------------------
% _ __  _ __ ___  ___ ___ _   _ _ __ ___ 
%| '_ \| '__/ _ \/ __/ __| | | | '__/ _ \
%| |_) | | |  __/\__ \__ \ |_| | | |  __/
%| .__/|_|  \___||___/___/\__,_|_|  \___|
%|_|                                     
% -------------------------------------------------------
\subsection{Pressure model}

\begin{tabular}{ |c|c|c| }
    \hline
    $p_A, p_B$              & $Pa$              & pressure in chamber A, B \\
    $\dot{m_A}, \dot{m_B}$  & $kg \cdot s^{-1}$ & mass flow on way to chamber A, B \\
    $S_A, S_B$              & $m^2$             & piston area  \\
    $V_A, V_B$              & $m^3$             & volume of chamber A,B \\
    $V_{0A}, V_{0B}$        & $m^3$             & "dead" volume of chamber A,B \\
    $m$                     & $kg$              & piston mass\\
    $F_{load}$              & $N$               & load \\
    $x$                     & $m$               & piston position \\
    $l$                     & $m$               & maximum piston position \\
    \hline
\end{tabular}

There are different approaches how to model thermal processes in pneumatic
system. Isothermal, adiabatic, polytropic models are suitable in different
technical applications. 


\paragraph{Isothermal model of pressure in cylinder}

\begin{align}
    m = \rho V \\
    \dot{m} = \dot{\rho} V + \rho \dot{V}
\end{align}

Applying \ref{eq:equation_of_state}:
\begin{align}
    \rho = \frac{p}{RT} \\
    \dot{\rho} = \frac{\dot{p}}{RT} 
\end{align}

Finally get \ref{eq:pressure1}:
\begin{align}
    \dot{p} = - \frac{p}{V}\dot{V} + \frac{RT}{V}\dot{m}
    \label{eq:pressure1}
\end{align}


\paragraph{Adiabatic model of pressure in cylinder} 
In this work adiabatic process was chosen with
respect to \ref{} \todo[inline]{Add source}.
For simple adiabatic model following equation can be used
\ref{eq:pressure_adiabatic_simple_model}:

\begin{align}
    \dot{p} = - \frac{\kappa p}{V}\dot{V} + \frac{\kappa RT}{V}\dot{m}
    \label{eq:pressure_adiabatic_simple_model}
\end{align}

\begin{align}
    \dot{p_A} = \frac{\kappa}{S_A x + V_{0A}} \left(- p_A S_A\dot{x} + RT_A\dot{m_A}
    \right)
\end{align}

\begin{align}
    \dot{p_B} = \frac{\kappa}{S_B (l-x) + V_{0B}} \left(p_B S_B\dot{x} + RT_B\dot{m_B}
    \right)
\end{align}

Volumes of chambers:
\begin{align}
    V_A = S_A x + V_{0A} \\
    V_B = S_B (l-x) + V_{0B} \\
    \dot{V}_A = S_A \dot{x} \\
    \dot{V}_B = - S_B \dot{x}
\end{align}

$T_A, T_B$ calculated from \ref{eq:equation_of_state}, or in adiabatic
model this parameters can remain constant same as atmospheric temperature.

\subsection{Mass flow model}

\subsubsection{Input/Output mass flows}

\begin{align}
    \dot{m}T = \dot{m_{in}}T_s - \dot{m_{out}}T_{A/B}
\end{align}


\subsubsection{Differential equation for Temperature change}
\begin{align}
    T = \int{ (\kappa T_s - T_{A/B})
        \frac{R\dot{m}_{A/Bin}}{p_{A/B}V_{A}}T_{A/B} -
    (\kappa-1)\frac{R\dot{m}_{A/Bout}}{p_{A/B}V_{A/B}} T_{A/B}^2 -
(\kappa-1)\frac{\dot{V}_{A/B}}{V_{A/B}}T_{A/B}}
\end{align}


\subsubsection{Valve model} %TODO
\begin{tabular}{ |c|c|c| }
    \hline
    $S_{eq}$                & $m^2$         & Equivalent cross section \\
    $S_{max}$               & $m^2$         & Maximum cross section \\
    $Cd$                    & $-$           & Coefficient of contraction \\
    $u$                     & $-$           & Regulation variable \\
    \hline
\end{tabular}

\paragraph{Valve flow model with simply input control signal}
For regulation flow this model used input control signal directly without
spool mechanics.

Coefficient of contraction \ref{eq:coefficient_of_contraction}:
\begin{align}
    C_d = \frac{S_{eq}}{S_{max}}
    \label{eq:coefficient_of_contraction}
\end{align}

For flow control regulation $u \in \langle-1,1\rangle$ can be used.
\begin{align}
    u =
    \begin{cases}
        u \in \langle -1, 0) & \text{discharge the chamber} \\
        u = 0& \text{valve closed}  \\
        u \in (0, 1\rangle & \text{filling the chamber} 
    \end{cases}
\end{align} 

\begin{align}
    \dot{m} = u S_{max} C_d p_1 \sqrt{\frac{2}{RT_1}}
    \cdot \psi\left(\frac{p_2}{p_1}\right)
    \label{eq:flow}
\end{align}

\textbf{For filling the chamber:}
\begin{itemize}
\item $p_1 = p_s$ 
\item $p_2 = p_A \text{ or } p_B$
\item $T_1 = T_s$
\end{itemize}

\textbf{For discharge the chamber:}
\begin{itemize}
\item $p_1 = p_A \text{ or } p_B$
\item $p_2 = p_0$
\item $T_1 = T_A, T_B$
\end{itemize}

where $p_s$ is supply pressure. $p_0$ atmospheric pressure. As $T_i$ - 
atmospheric temperature using according to isothermal process.

\begin{align}
    \dot{m}_A =
    \begin{cases}
        u S_v C_d p_s \sqrt{\frac{2}{RT_s}}
        \cdot \psi\left(\frac{p_A}{p_s}\right)  &,   u \in (0, 1 \rangle \\
        0   &,  u = 0 \\
        u S_v C_d p_A \sqrt{\frac{2}{RT_A}}
        \cdot \psi\left(\frac{p_0}{p_A}\right)  &,   u \in \langle -1, 0) \\
    \end{cases}
\end{align}

\begin{align}
    \dot{m}_B =
    \begin{cases}
        u S_v C_d p_s \sqrt{\frac{2}{RT_s}}
        \cdot \psi\left(\frac{p_B}{p_s}\right)  &,   u \in (0, 1 \rangle \\
        0   &,  u = 0 \\
        u S_v C_d p_A \sqrt{\frac{2}{RT_B}}
        \cdot \psi\left(\frac{p_0}{p_B}\right)  &,   u \in \langle -1, 0) \\
    \end{cases}
\end{align}

\paragraph{Valve flow with spool mechanic included}

With respect to valve spool modeled as 1DOF system \ref{eq:1dof} and
mechanical and geometrical properties following equation were used.

\paragraph{Valve flow with spool}
In this model we accept a spool displacement $x_s$, controlled by input
voltage $u$.

\begin{equation}
    \dot{m}(P_u, P_d) = 
    \begin{cases}
        C_f A_v
        \left(\frac{\kappa}{R}\left(\frac{2}{\kappa-1}\right)\right)^{\frac{1}{2}}
        \cdot
        \frac{P_u}{\sqrt{T}}\left(\frac{P_d}{P_u}\right)^{\frac{1}{\kappa}}
        \cdot 
        \sqrt{1 - \left(\frac{P_d}{P_u}\right)^{\frac{\kappa-1}{\kappa}}} &,
            \text{ if } \frac{P_d}{P_u}>P_{cr} \text{ (subsonic)} \\
        C_f A_v \frac{P_u}{\sqrt{T}}\cdot \sqrt{\frac{\kappa}{R}
        \left(\frac{2}{\kappa + 1}\right)^{\frac{\kappa+1}{\kappa-1}}} &,
            \text{ if }  \frac{P_d}{P_u} \le P_{cr} \text{ (sonic)} \\
    \end{cases}
    \label{eq:valve_2}
\end{equation}

where $C_f$ is discharge coefficient, $A_v$ is the effective are of valve
orifice.

\begin{equation}
    A_v = \frac{\pi x_s^2}{4}
    \label{eq:A_v}
\end{equation}

\begin{equation}
    x_s = C_v u
    \label{eq:x_s}
\end{equation}

where $C_v$ is the valve constant.

\paragraph{Valve model by Endler}
Require fitting constants and generally system identification.
Mass flow rates are given by following equations:


\begin{equation}
    \begin{aligned}
        \dot{m}_A(u, p_A) = g_1(p_A, sign(u))arctg(2u) \\
        \dot{m}_B(u, p_B) = g_2(p_B, sign(u))arctg(2u)
    \end{aligned}
\end{equation}

where $g_1, g_2$ are signal functions given:
\begin{equation}
    \begin{aligned}
        g_1(p_A, sign(u)) = \beta \Delta p_A = 
        \begin{cases}
            (p_s - p_A) \beta^{ench} &, \rm \ if\  u \ge 0 \\
            (p_A - p_0) \beta^{esv} &, \rm \ if\  u  < 0 \\
        \end{cases} \\
        g_2(p_B, sign(u)) = \beta \Delta p_B = 
        \begin{cases}
            (p_s - p_B) \beta^{ench} &, \rm \ if\  u < 0 \\
            (p_B - p_0) \beta^{esv} &, \rm \ if\  u \ge 0 \\
        \end{cases}
    \end{aligned}
    \label{eq:valve_3}
\end{equation}

where $\beta^{ench}, \beta^{evs}$ are constant coefficients.
For fitting model stop piston (speed of piston is null). This mean that
volume is constant. We can measure flow rate $\dot{m}$ versus input voltage
$u$ with given pressure difference.

\paragraph{Valve dead-zone}
For more precision control and modeling of the valve system, valve
dead-zone can be used \ref{eq:deadzone}.

\begin{equation}
    u_z = 
    \begin{cases}
        g_z(u) < 0 &, \text{ if } u \le u_n \\
        0          &, \text{ if } u_n < u < u_p \\
        h_z(u) > 0 &, \text{ if } u \ge u_p \\
    \end{cases}  
    \label{eq:deadzone}
\end{equation}



% -------------------------------------------------------
% _ __ ___   ___  ___| |__   __ _ _ __ (_) ___ 
%| '_ ` _ \ / _ \/ __| '_ \ / _` | '_ \| |/ __|
%| | | | | |  __/ (__| | | | (_| | | | | | (__ 
%|_| |_| |_|\___|\___|_| |_|\__,_|_| |_|_|\___|
% -------------------------------------------------------
\subsection{Mechanical assembly}
\subsubsection{Equation of motion}

The motion of the pneumatic piston mechanism describes in terms of the
general 1dof dynamical equation \ref{eq:1dof}. 

\begin{equation}
    m\ddot{x} + b\dot{x} + kx = u
    \label{eq:1dof}
\end{equation}

In the case of the pneumatic piston, the equation \ref{eq:1dof}
transforms into an equation \ref{eq:mechanical}.

\begin{equation}
    (M + M_L) \ddot{x} + F_{damp} + F_g + F_{hs}  = F_p
    \label{eq:mechanical}
\end{equation}

Where $M$ represents a mass of the all moveable part of the piston,
$M_L$ is load mass, $F_g$ gravity force acting to mechanical moving assembly,
$F_{hs}$ - models endpoints (hard stop),
$F_{damp}$ represents shock absorbers acted at endpoints,
$F_{p}$ is a force produced by the pneumatic piston \ref{eq:pneum}.

\begin{equation}
    F_p = P_A S_A - P_B S_B - P_0 S_0
    \label{eq:pneum}
\end{equation}

\subsubsection{Hard stop}
Hard stop can be represented as spring and dumps:

\begin{align}
    F_{HS} =
    \begin{cases}
        K_p(x-g_p) + D_pv & \text{for } x \ge g_p \\
        0 & \text{for } g_n < x < g_p \\
        K_n(x-g_n) + D_nv & \text{for } x \le g_n \\
    \end{cases}
\end{align}


\subsubsection{Shock Absorbers}
\subsubsection{Friction}
Friction force can be modeled in the
different ways.

As an example of possible model is following equation. That consist from
complex friction forces including viscous friction and Coulomb friction
\ref{eq:friction1}.
\begin{equation}
    F_f = 
    \begin{cases}
        C \dot{x} + \left(f_c + (f_s-f_c)
        e^{-\left(\frac{\dot{x}}{v_s}\right)^{\delta}}\right) sign(\dot{x}) &,
        \text{ if } \dot{x} \le v_e \\
        \mu \dot{x} &,
        \text{ if } \dot{x} > v_e \\
    \end{cases}
    \label{eq:friction1}
\end{equation}
where $C$ - viscous friction coefficient, $f_c$ - Coulomb friction, $f_s$ -
maximum static friction, $\mu$ - dynamic friction factor, $v_s$ - Stribeck velocity,
$\delta$ - arbitrary index, $v_e$ critical velocity.


% -------------------------------------------------------
% ___  ___ _ __  ___  ___  _ __ ___ 
%/ __|/ _ \ '_ \/ __|/ _ \| '__/ __|
%\__ \  __/ | | \__ \ (_) | |  \__ \
%|___/\___|_| |_|___/\___/|_|  |___/
% -------------------------------------------------------
\subsection{Sensors Modeling}
Sensors

% -------------------------------------------------------
%(_) __| | ___ _ __ | |_ 
%| |/ _` |/ _ \ '_ \| __|
%| | (_| |  __/ | | | |_ 
%|_|\__,_|\___|_| |_|\__|
% -------------------------------------------------------
\subsection{Parameter identification}

\subsubsection{Mechanical assembly}
In mechanical system there is $F_f$ force represented by frictions accruing
in the system. This force can be modeled by different friction models with
respect to \ref{sec:mech_assembly}. Friction force parameters can be
estimated using "gray-box" method. 
Using $\dot{m}$ mass flow data versus $x$ position measured on real assembly
and use these data as an input and output, we can fit $F_f$.
Simplify model can contain TODO:
\begin{itemize}
    \item $F_C$ static friction
    \item $C_v$ viscous
    \item $C_p$ Pressure difference
\end{itemize}

\subsubsection{Cylinder}
Dead volume: $p_1 V_1^n = p_2 V_2^n$ or datasheet.

\subsubsection{Valve}
For valve system there are two parameters that need to be estimated.
According to equation \ref{eq:flow2} with constant $p_1$ (pressure supply) and $p_2$
(atmospheric pressure), we can estimate $\boldsymbol{C}$ if we neglect Valve Spool dynamic.
If in experiment we determine that spool dynamic necessary to include. We
provide same experiment with spool model including to "Gray-box" fitting
model.
\begin{align}
    \dot{m} = \boldsymbol{u}(x_s) \boldsymbol{C}  p_1 \sqrt{\frac{2}{RT_1}}
    \cdot \psi\left(\frac{p_2}{p_1}\right)
    \label{eq:flow2}
\end{align}


% -------------------------------------------------------
%  __ _ _ __  _ __  _ __ _____  __
% / _` | '_ \| '_ \| '__/ _ \ \/ /
%| (_| | |_) | |_) | | | (_) >  < 
% \__,_| .__/| .__/|_|  \___/_/\_\
%      |_|   |_|                  
% -------------------------------------------------------

\section{Alternative Modeling Approaches (3 pages)}
Generally with dataset of input-output signals approximation model can be
fit. Using System Identification Toolbox and modeled as Black-Box or
Gray-Box models. This section attempted to fit some models using data from
SimScape and Equation model presented before.

Fit approximation model make sense only if we know what to fit. Using
signal process techniques and identify dominant signals that providing best
classification features we will train models with respect to this signals.


\subsection{Simscape Model}
Working, very slow.


\subsection{State-space Model}
Not working, Nonlinearities.

\subsection{ARX Model}
Not working, Nonlinearities.

\subsection{Hammerstein-Wiener Model}
Working only for position.

\subsection{Nonparametric Model}
Working.


% -------------------------------------------------------
%  ___ ___  _ __ ___  _ __   __ _ _ __ ___ 
% / __/ _ \| '_ ` _ \| '_ \ / _` | '__/ _ \
%| (_| (_) | | | | | | |_) | (_| | | |  __/
% \___\___/|_| |_| |_| .__/ \__,_|_|  \___|
%                    |_|                   
% -------------------------------------------------------
\section{Models comparison (2-3 pages)}
\subsection{Model based on equations}
This model \ref{fig:model_equations} was developed with respect to equations
represented before.

\begin{figure}[h!]
    \centering
    \includegraphics[width=1\textwidth]{equations.png}
    \caption{Simulink model based on equations}
    \label{fig:model_equations}
\end{figure}


\subsection{Model Simscape}
Model \ref{fig:model_simscape} was developed using SimScape toolbox.

\begin{figure}[h!]
    \centering
    \includegraphics[width=1\textwidth]{simscape.png}
    \caption{Simulink model using SimScape Toolbox}
    \label{fig:model_simscape}
\end{figure}

\subsection{Nonparametric model}



\subsection{Comparison}
Following figure \ref{fig:compare_of_models} represent comparison of 2 models
(Simscape and based on equations) using same parameters for simulation:
There is slight difference between models causing Valve dynamics
simplifications in model based on equations.

\begin{figure}[h!]
    \centering
    \includegraphics[width=1\textwidth]{models_comparation.png}
    \caption{Comparison of simscape and model based on equations}
    \label{fig:compare_of_models}
\end{figure}

\end{document}


\chapter{Alternative Modeling Techniques}\label{ch:alt_model}

This chapter deals with other possibilities of modeling the technical
system, particularly the double-acting pneumatic piston. Physical modeling
and data-driven modeling methods were examined in terms of suitability for
applying FDI and PdM strategies.

\section{Physical Modeling}
Physical modeling operates with models with a compiled layout that matches
the structure of the different physical domains. In this type of software,
it is possible to combine different domains to create a complex system
model.

Matlab/Simulink provides a physical modeling library, Simscape
\cite{simscape}, that meets
the above specifications. Using Simscape software, the user combines a
model from different blocks representing different physical functions
(spring, resistance, hydraulic valve), and connection links represent some
types of energy flow.



\subsection{The double-acting pneumatic piston modeling in Simscape}

In this part, the same assumption applies as in section \ref{assumptions}. All the
processes take place adiabatically, i.e., without heat exchange with the
environment. 

The resulting model was compiled using gas and mechanical domains \ref{fig:simscape}.

\begin{figure}[h!]
    \centering
    \includegraphics[width=1\textwidth]{simscape.png}
    \caption{The double-acting pneumatic piston developed using Simscape
    software}
    \label{fig:simscape}
\end{figure}


\subsection{Limitations}

It is necessary to know well the parameters of the system.

For example, we need to have a precision-measured characteristic of flow
control valve adjustment in the form of a lookup table to use a throttle
valve block.

Providing simplification and reduce the model to the only control valve,
there are still a few parameters that are not available such as valve and
dampers coefficients mentioned before.

The main problem is the computational complexity of the model compared with
the first principle model. During the parameter estimation, the first
principle model is much faster than the Simscape model and gives an option
to experiment with different fault states analysis. 

However, both models showed quite close behavior during testing with the
same parameters.

\section{Data-Driven Models}
Data-Driven modeling explores collected measured signals to identify the
system structure or learn the system behavior from data \cite{ident}.

Between data-driven common models belongs parametric and non-parametric
models. Parametric models take part in the system identification field. A
collection of different generalized mathematical models can be fitted to
the input-output signals pair, such as transfer functions, polynomial
models, non-linear ARX models, etc.   A typical representative of
non-parametric models are neural networks of various structures.  In this
thesis, experiments on test datasets were performed with both types of
models. 

\subsection{Hammerstein-Wiener Model}

\begin{figure}[h!]
    \centering
    \includegraphics[width=0.7\textwidth]{hw_model.pdf}
    \caption{Hammerstein-Wiener model structure}
    \label{fig:hw_model}
\end{figure}

The best results between parametric models using System Identification
Toolbox, shown Hammerstein-Wiener Model. The model consists of three blocks
\ref{fig:hw_model}, input nonlinearity, linear block and output
nonlinearity.  The nonlinearities are represented by different functions
such as dead-zone, polynomial estimator, saturation, wavelet network
function, etc. 

\begin{figure}[h!]
    \centering
    \includegraphics[width=0.8\textwidth]{hw_position.png}
    \caption{Simulated Response for Position Signal Comparison}
    \label{fig:hw_position}
\end{figure}

\begin{figure}[h!]
    \centering
    \includegraphics[width=0.8\textwidth]{hw_velocity.png}
    \caption{Simulated Response for Velocity Signal Comparison}
    \label{fig:hw_velocity}
\end{figure}

However, using the identified model, adequate behavior to the measured data
was achieved only for the position signal \ref{fig:hw_position}. The model identified
for velocity signal did not show acceptable behavior \ref{fig:hw_velocity}.  The
reason is the significant nonlinearity and complexity of the system, which
the simplified models cannot take into account.



\newpage
\subsection{NARX Model}

Different structures can be used to train the neural network to predict
system behavior. The most common way is using the nonlinear autoregressive
with the external input model (NARX) \cite{ident}. This model predicts time-series data
by using different numbers of time-delayed values of input and output
signals \ref{fig:ann}.

\begin{figure}[h!]
    \centering
    \includegraphics[width=0.5\textwidth]{ann.pdf}
    \caption{Schematical representation of NARX model}
    \label{fig:ann}
\end{figure}


During the development of the model, it is necessary to pay attention to
overfitting, which can significantly impair the performance of the model
and its generalization capabilities.

Some experiments have been performed with this modeling approach. The
Neural Network can predict the behavior of the system based on input.

% -------------------------------------------------------
%  ___ ___  _ __ ___  _ __   __ _ _ __ ___ 
% / __/ _ \| '_ ` _ \| '_ \ / _` | '__/ _ \
%| (_| (_) | | | | | | |_) | (_| | | |  __/
% \___\___/|_| |_| |_| .__/ \__,_|_|  \___|
%                    |_|                   
% -------------------------------------------------------
\chapter{Models comparison (2-3 pages)}


\section{First Principle Model}
This model \ref{fig:model_equations} was developed with respect to equations
represented before.

\begin{figure}[h!]
    \centering
    \includegraphics[width=1\textwidth]{equations.png}
    \caption{Simulink model based on equations}
    \label{fig:model_equations}
\end{figure}

\section{Alternative Modeling Techniques (3 pages)}
Generally with dataset of input-output signals approximation model can be
fit. Using System Identification Toolbox and modeled as Black-Box or
Gray-Box models. This section attempted to fit some models using data from
SimScape and Equation model presented before.

Fit approximation model make sense only if we know what to fit. Using
signal process techniques and identify dominant signals that providing best
classification features we will train models with respect to this signals.

\subsection{Physical Model (SimScape)}
Working, very slow. Equations are faster for estimation parameters.
Model \ref{fig:model_simscape} was developed using SimScape toolbox.

\begin{figure}[h!]
    \centering
    \includegraphics[width=1\textwidth]{simscape.png}
    \caption{Simulink model using SimScape Toolbox}
    \label{fig:model_simscape}
\end{figure}

\subsection{State-space/ARX Models}
Not working, Nonlinearities.


\subsection{Hammerstein-Wiener Model}
Working only for Position.

\subsection{Nonparametric model (ANN)}

Working. Can be used as "Normal operation" model.



\section{Comparison}
Following figure \ref{fig:compare_of_models} represent comparison of 2 models
(Simscape and based on equations) using same parameters for simulation:
There is slight difference between models causing Valve dynamics
simplifications in model based on equations.

\begin{figure}[h!]
    \centering
    \includegraphics[width=0.8\textwidth]{models_comparation.png}
    \caption{Comparison of simscape and model based on equations}
    \label{fig:compare_of_models}
\end{figure}



% Artyom Voronin
%     _                     
% ___| |__  _ __  _ __ ___  
%/ __| '_ \| '_ \| '_ ` _ \ 
%\__ \ |_) | |_) | | | | | |
%|___/_.__/| .__/|_| |_| |_|
%          |_|              
%
% Brno, 2021


% ----------------------------------------------------------------------------- 

% ----------------------------------------------------------------------------- 

\chapter{Signal-Based PdM}
Signal-based predictive maintenance.
\section{Data Management and Preprocessing}
Before the final solution was developed in the whole dataset, the smaller
the reduced dataset was used for experiments and planning algorithms. 

\subsection{Data storage}
\paragraph{Manage Data}
First, a folder structure was created to collect all measured and
calculated data. The measured signals were given in 6 large files with a
".mat" extension and divided into smaller files with only one measurement
each.  The divided datastore is easier to maintain and Matlab recommends
this type of datastores called Data Ensemble \ref{}.  The full dataset
contains 4840 measurements. Each measurement includes a 10-second recording
of all signals collected from moving the piston up and down.
\paragraph{Labels}
The whole dataset was divided into 20 Labels by place of fault accumulate: 
\begin{itemize}
    \item Healthy
    \item Throttle valve 1
    \item Throttle valve 2 
    \item Small damper bottom
    \item Small damper upper
    \item Large dampers 
    \item And combinations of these faults
\end{itemize}

\subsection{Data Exploration}
Data from each sensor were explored in an attempt to find measurement
errors or anomalies in data.  Figure \ref{} shown an example of the
flow signal in different operation conditions. 

There are eight types of sensors:
\begin{enumerate}
    \item Linear encoder
    \item Flow sensor
    \item Pressure sensor
    \item Temperature sensor
    \item Accelerometer
    \item Strain gauge
    \item Microphones
    \item Proximity sensors
\end{enumerate}

\subsection{Preprocessing}
Some signals, such as Encoder, are very accurate. There is no preprocessing
needed to apply. Signals noisier (pressure signal or strain) has to be
preprocessed and applied some noise reduction algorithms. However, during
experiments turned out that non preprocessed signals have better
performance. For example, the preprocessed pressure classification model
gives 78 \% accuracy; the raw pressure signal gives approximately 82 \%.

\section{SB methods and Flow Sensor as an Example}
In this section, signal-based methods were applying to the flow sensor as a
case study example.  The rest of the sensors was processed in the same way;
however, each required an individual approach.

\subsection{Flow Sensor Data}
There are two flow signals in the datastore. Both are connected to port A
in scheme \ref{}.
\begin{itemize}
    \item Flow Extrusion
    \item Flow Contraction
\end{itemize}

\subsection{Condition Indicators Extraction}
\paragraph{Statistical Condition Indicators}
From every signal in the dataset, statistical features were calculated.

\paragraph{Frequency Domain Condition Indicators}
Using Welch's power spectral density estimation, frequency features were
calculated. 


Extracted condition indicators were written to files with signals and
easily acceptable. After each data file contains complete information about
one measurement:
\begin{itemize}
    \item Measured signals
    \item Setting parameters (valves, dampers, load)
    \item Power spectrum calculated from measured signals
    \item Statistical and Frequency features extracted from signals
\end{itemize}

Moreover, a table was created, which contains all condition indicators
extracted, to prepare the train and test dataset for the classification
model.

\subsection{Features Ranking}
Kruskal-Wallis ANOVA

\subsection{Train Classification Model}

\paragraph{Split CI to train and test}

\paragraph{Classification Model Performance}

\section{Summary All Sensors Comparison}




%Dataset was divided to 5 main categories.
%
%
%Data has been accumulated to ".mat" files.
%Each file contains signals from sensors during 10 seconds measurements with
%different pneumatic actuator configuration. Example results from one
%experiment are represented in figures \ref{fig:data_exmp1},
%\ref{fig:data_exmp2}. 
%
%
%\section{Data management}
%Before the final solution was developed in the whole dataset, the smaller
%reduced dataset was used for experiments and planning algorithms. 
%
%\paragraph{Data Ensembles}
%Data files have been reshaped to Data Ensembles format used for Condition
%monitoring purposes. This format allows processing data without copying the
%whole dataset to memory at once but processes them one by one. In large datasets
%it gives an option to manipulate with data without problems with allocated memory.
%
%Divided to 3 datasets:
%\begin{itemize}
%    \item Train data
%    \item Validation data
%    \item Test data
%\end{itemize}
%
%\section{Preprocessing}
%Measured signals require preprocessing concerning the preservation of the information
%base. For smoothing data Moving Average function were used.
%As an example, the figure \ref{fig:preprocess} is shown the "raw" and filtered signals.
%The whole dataset of preprocessed data is relatively big. For
%time-saving, parallel computing was used for all computationally
%demanding parts of the code.
%
%
%
%\section{FDI methods}
%
%\subsection{Line checking}
%
%We can use Proximity sensor time delay between input signal and upper
%proximity sensor signal to evaluate if there is some fault.
%
%Same with Position, if not reach some end position, there is a fault.
%
%Flow sensor, check if the float mean value is under some threshold, there
%is fault.
%
%
%\section{Condition Indicators extraction}
%
%For classification task purpose from the signals have been extracted
%statistical features such as mean, median, peak to peak value, etc.
%As a condition "FaultCode" variable
%were used. This variable represent configuration of pneumatic actuator
%during the measurement.
%
%All calculated features were added to the dataset and were ranked by
%Kruskal-Wallis ANOVA algorithm. Following table \ref{tab:feat} contain
%5 first best features ranked for classification purpose.
%
%Kruskal-Wallis is very suitable to ranking features before using PCA or
%SVD.
%
%\paragraph{Selecting Condition Indicators} There is a problem if we will
%deploy classification task with large features dataset.
%There are different possibilities to reduce data before train
%classification model or do a prediction. On of them is to rank a features
%by Analysis of Variation algorithm to evaluate a good representation
%features.
%
%
%% PCA vs Sort features(Anova) 
%
%
%\subsection{Microphones}
%Cheap, good results, but maybe problems with real life integration (noise
%from another machines). Another problem cannot be modeled in simulation
%system. For predictive purposes require data from real model.
%
%\subsection{Encoder}
%Good results, useful in simulations and compare results with Digital Twin.
%Can be used in Model-Based CI. 
%Digital twin can generate fault data, that will be applicable with encoder
%sensor.
%
%\subsection{Acceleration sensors}
%Not good, not bad. Can be used for classification task. But encoder has
%more accuracy information.
%
%\subsection{Proximity Sensors}
%Cheap. Very correlated features.Can not be used for classification. But suitable
%to detect binary classification (Health, Failed).
%Only statistical features, no Frequency domain.
%
%\subsection{Flow Sensors}
%Very expensive sensors. Not so good results.
%
%\subsection{Air Pressure}
%This sensor always used, to control pressure valve. But not good results.
%Maybe in combination with another sensor.
%
%\subsection{Strain Gauge}
%Expensive, Normal results of classification. But not suitable for
%Simulation Model.
%
%\subsection{Temperature}
%Good results on data. But only because Ambient temperature was changed
%between measurements. In one day it was warm, another colder :)
%
%
%\section{Classification Task}
%
%The main goal of the classification task is to train a model that can
%predict the "FaultCode", or "Label" signalized about pneumatic actuator behavior by
%calculated features.
%
%Using Kuskal-Wallis one way analysis of variance, features were ranked by
%importance with respect to correlation. This gives opportunity to reduce
%number of features before PCA analysis.
%
%Principal component analysis (PCA) has been used to reduce the number of
%features and chose the best representants.
%
%The trained model has been exported to \textbf{models/} directory.
%

% Artyom Voronin
%           _                     
% _ __ ___ | |__  _ __  _ __ ___  
%| '_ ` _ \| '_ \| '_ \| '_ ` _ \ 
%| | | | | | |_) | |_) | | | | | |
%|_| |_| |_|_.__/| .__/|_| |_| |_|
%                |_|              
%
% Brno, 2021

\documentclass[class=article, crop=false]{standalone}
\usepackage[subpreambles=true]{standalone}
\usepackage{subcaption}

\usepackage{xargs}
\usepackage[pdftex,dvipsnames]{xcolor}  % Coloured text etc.
\usepackage[colorinlistoftodos,prependcaption,textsize=tiny]{todonotes}

\usepackage{sectsty}
\usepackage{graphicx}
\graphicspath{{img/}{../img/}{../../img/}}
\usepackage{listings}
\lstset{language=Matlab}
\usepackage{hyperref}
\usepackage{amsmath}
\usepackage{import}
\usepackage{subfiles}
\usepackage{caption}
\usepackage[utf8]{inputenc}
\usepackage[english]{babel}

%\usepackage[square, numbers]{natbib}
%\bibliographystyle{unsrtnat}
%\usepackage[nottoc]{tocbibind}

\topmargin=-0.45in
\evensidemargin=0in
\oddsidemargin=0in
\textwidth=6.5in
\textheight=9.0in
\headsep=0.25in

%\title{Preprocessing data from pneumatic actuator}
%\author{Artyom Voronin} 
%\date{}

\begin{document}
%\tableofcontents


\section{PdM using a Simulation Model}

\subsection{Differences between Model-Based PdM and PdM using Digital Twin}
There is a difference between using Model-Based PdM and using Simulation
Model as a Digital Twin.

\subsection{Model-Based Condition Indicators}
Model-Based approach is suitable when it's difficult to identify condition
indicators using only signals. In some cases it's useful to fit some model
from data and extract condition indicators as some system parameter.

\subsubsection{Static and Dynamic Models}
If the system behavior can be fit from the data as a static model, than we
can extract condition variables from this model. For example, if model
was fitting to a polynomial model, than polynomial coefficients can be use
as condition indicators.

Signals showing dynamic behavior can be fitted to dynamic models such as
State-Space or AR, ARX, NLARX (Nonlinear auto recursive model) and so on.
Then condition indicators can be extracted as poles, zeros damping
coefficients from estimated model.


\subsection{Using Simulation Model for Residuals
Estimation}\label{sec:residuals}
Another option is using the Simulink model with \textbf{prediction
error minimization function} to compute difference between Simulink model
and measured data. From this difference we can separate fault condition and
healthy operation. 


\subsubsection{Comparison with Nominal System Model}
\todo[inline]{Same thing as section \ref{sec:residuals}}

Compare actual system behavior with system model. This will generate some
error $e(t) = y(t) - \hat{y}(t)$. From this error residual can be generated
in form $r(t)=\Phi(u_t,y_t, \varepsilon_t,v_t,d)$ and after some decision.

\subsection{Using Digital Twin to Generate Fault Data}

\subsection{Using Digital Twin to Generate Prognostic Data}

\subsection{RUL}


\end{document}

\chapter{Conclusion}

\section{Simulation Model}
One of the outcomes from the thesis is a simulated model built based on
differential equations from the pneumatic-mechanical domain, modeled and
developed using Matlab/Simulink software. The simulation model was
estimated with parameters of healthy system behavior. However, there is an
option to reestimate parameters to fault state and simulate the system in a
fault condition. 

Due to the available measured data and significantly nonlinear dynamics of
the system, the simulation model shows good agreement with the measured
data. In contrast to the model built using SimScape library, it is
distinctly less computationally expensive while maintaining numerical
stability. These facts are fundamental when parameter estimation is in
progress.

The simulation model can be used to experiment with the system's behavior
in different conditions, model fault situations and generate data for the
design and development of the robust predictive maintenance algorithms. 


\section{Signal-Based PdM}
Another outcome is verifying the possibility of classification and
detection of a fault condition applying predictive maintenance techniques,
using signal-based and model-based methods.

The experiments were performed on a dataset measured on a demonstration
device using seven types of sensors.
  

A signal-based method is based on the extraction of useful information
directly from the signal in time-frequency domains. Each sensor required an
individual approach for preprocessing, extracting features, ranking
features, and building the classification models. But generally, there is
minimal preprocess needed to keep the possible helpful information. 

The table \ref{tab:sensors_final} contains the comparison of sensors in 2 categories, accuracy
performed in the test dataset and sensor cost. The graph
\ref{fig:sensors_final_bar} visualizes these data.

Surprisingly, all sensors showed an accuracy of more than 75\%. Microphones
offer excellent performance from a cost/accuracy perspective, and they are
suitable for installation and maintenance.

\begin{figure}[h!]
    \centering
    \includegraphics[width=1\textwidth]{sensors_final_bar.png}
    \caption{Comparison of sensors from accuracy/cost perspective}
    \label{fig:sensors_final_bar}
\end{figure}

\begin{table}[h]
    \centering
    \begin{tabular}{|c|c|c|c|c|c|c|c|}
        \hline
        \textbf{Sensor}   & Acc & Encoder & Flow & Mics & Pressure & Proximity & Strain \\
        \hline
        \textbf{Accuracy [\%]} & 91.6 & 96.1 & 97.2 & 95.8 & 76.6 & 80.5 & 95.0 \\
        \hline
        \textbf{Cost [czk]} & 2x 3500 & 25000 & 6000 & 3x 500 & 1000 & 2x 1000 & 15000 \\
        \hline
    \end{tabular}
    \caption{Comparison of sensors from accuracy/cost perspective}
    \label{tab:sensors_final}
\end{table}

\section{Model-Based PdM}
The next part of this thesis was to apply model-based methods and using a
simulation model for predictive maintenance algorithms. These algorithms
are practical when it's hard to extract useful information using a
signal-based method. Or it's suitable in some cases where we understand
the system dynamics and know how to exploit some system variables as
condition indicators.

The use of the method of extraction features in the form of a Nonlinear
system identification model coefficient, specifically with the
Hammerstein-Wiener model, did not give reliable results. Extracted features
have no statistical addiction, and it's impossible to predict fault type
using this method on the measured data from the pneumatic piston as a case
study.

On the other hand, the residual estimation using the simulation model
showed excellent results. The measured position signal was compared with
the signal from the simulation model. This residual signal was used to
classify the fault condition and achieve  99 \% on a smaller dataset.  But
given the results obtained using the signal-based method, the residual
estimation method may seem unnecessary. In this particular case, from a
practical point of view, the improvement of the result by a few percent
does not bring fundamental changes, but the calculation time increases
significantly. 

The possibility of modeling and simulation sensor faults was also verified
using the simulation model. Although it is difficult to collect data from
the sensor fault in real-life conditions, fault data can be generated from
the simulation model and even combined with the primary dataset to create a
synthetic dataset.

\subsection{RUL}
One of the main goals of predictive maintenance is to estimate the
remaining useful life. The original dataset does not contain a record of
historical data that shows degradation behavior. 

A common problem in the maintenance of pneumatic actuators is the leakage
of air from the chamber by the piston. This situation was modeled on the
simulation model and generated data used for RUL estimation. 

In the demonstration example, a flow signal was measured. From the
measurements, the shape factor parameter was calculated and used as a
condition indicator. The generated dataset contains 25 simulations with
different failure dynamics. Each measurement includes various 10 seconds
cycles, depending on the failure dynamic, before the system failure occurs.
The outcome is that it is possible to estimate the remaining useful life on
generated degradation dataset by using the residual similarity model,
pairwise similarity model, and linear degradation model. The prediction
results are satisfying.

\section{Further Development}
As a further development, it would be appropriate to estimate the modeled
system parameters piecewise to improve the results, emphasizing the
characteristics of throttle valves and dampers with adjustments. 

Perform air leak fault condition measurements and collect historical
degradation data from a real pneumatic piston. Subsequently, evaluate the
dynamics of the failure caused by the air leak. Verify the possibility of
estimating the remaining useful life using a flow sensor. An interesting
case study could be to verify if it is possible to estimate RUL using
microphones.

If the performance of the available sensors will be deficient, the pressure
measurements in the chamber can be performed. An example from the
simulation model is in section TODO.




\chapter*{List of Abbreviations}
\addcontentsline{toc}{chapter}{List of Abbreviations}
\label{chap:abb}

\begin{itemize}[leftmargin=2.7cm]
	\item[\textbf{LWL}] Locally Weighted Learning
	\item[\textbf{LS}] Least Squares Method
	\item[\textbf{RLS}] Recursive Least Squares Method
	\item[\textbf{RFWR}] Receptive Field Weighted Regression
	\item[\textbf{LOLIMOT}] Local Linear Model Tree
	\item[\textbf{EGR}] Exhaust Gas Recirculation
	\item[\textbf{PID}] Proportional-Integrational-Derivative controller
	
	
\end{itemize}

%\listofffigures
%\listofftables

\bibliography{citations.bib}
\bibliographystyle{plain}


\end{document}
