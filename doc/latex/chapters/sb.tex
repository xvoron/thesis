% Artyom Voronin
%     _                     
% ___| |__  _ __  _ __ ___  
%/ __| '_ \| '_ \| '_ ` _ \ 
%\__ \ |_) | |_) | | | | | |
%|___/_.__/| .__/|_| |_| |_|
%          |_|              
%
% Brno, 2021


% ----------------------------------------------------------------------------- 

% ----------------------------------------------------------------------------- 

\chapter{Signal-Based PdM}
Signal-based predictive maintenance.

\todo[inline]{Text about signal-based PdM}


\section{FDI methods}
\todo[inline]{More examples, rewrite}
We can use Proximity sensor time delay between input signal and upper
proximity sensor signal to evaluate if there is some fault.
Same with Position, if not reach some end position, there is a fault.
Flow sensor, check if the float mean value is under some threshold, there
is fault.

\section{Data Management and Preprocessing}
Before the final solution was developed in the whole dataset, the smaller
dataset was used for experiments and planning algorithms. 

\subsection{Data Storage}
\paragraph{Manage Data}

First, a folder structure was created to collect all measured and
calculated data. The measured signals were given in 6 large files with a
".mat" extension and divided into smaller files with only one measurement
each.  Data files have been reshaped to Data Ensembles \cite{} format used
for Condition monitoring purposes. This format allows processing data
without copying the whole dataset to memory at once but processes them one
by one. In large datasets it gives an option to manipulate with data
without problems with allocated memory.  The full dataset contains 4840
measurements. Each measurement includes a 10-second recording of all
signals collected from moving the piston up and down.

\paragraph{Labels}
The whole dataset was divided into 20 Labels by place of fault accumulate: 
\begin{itemize}
    \item Healthy
    \item Throttle valve 1
    \item Throttle valve 2 
    \item Small damper bottom
    \item Small damper upper
    \item Large dampers 
    \item And combinations of these faults
\end{itemize}

\subsection{Data Exploration}
Data from each of the eight sensors \ref{tab:sensors_tab} were explored in an attempt to find
measurement errors or anomalies in data.  Figure \ref{fig:flow_sig} shown
an example of the flow signal in different operation conditions. 

\begin{figure}[h!]
    \centering
    \includegraphics[width=1\textwidth]{sb_flow_signal.png}
    \caption{Flow Signal in Different Operation Conditions}
    \label{fig:flow_sig}
\end{figure}


%There are eight types of sensors:
%\begin{enumerate}
%    \item Linear encoder
%    \item Flow sensor
%    \item Pressure sensor
%    \item Temperature sensor
%    \item Accelerometer
%    \item Strain gauge
%    \item Microphones
%    \item Proximity sensors
%\end{enumerate}

\subsection{Preprocessing}

After the data has been processed and organized in one datastore, the
possibility arises to perform signal preprocessing.  Preprocessing includes
smoothing, filtering, detrend the signal, and missing value removal.

The datastore contains some signals, such as an encoder, that is very
accurate. There is no preprocessing needed to apply. Signals noisier
(pressure signal or strain) have to be preprocessed and applied algorithms
to noise reduction such as smoothing and filtering concerning the
preservation of the information base. However, during experiments turned
out that non preprocessed signals have better performance. For example, the
preprocessed pressure classification model gives 78 \% accuracy; model
trained on CI from the raw pressure signal offers approximately 82 \%.

\section{SB Methods and Flow Sensor as an Example}
In this section, signal-based methods were applying to the flow sensor as a
case study example.  The rest of the sensors was processed in the same way;
however, each required an individual approach.

\subsection{Flow Sensor Data}
There are two flow signals in the datastore. Both are connected to port A
in scheme \ref{}. Signals were sampled in 1kHz frequency; thus, in 10
seconds, there are 10000 points measured. 

\begin{itemize}
    \item Flow Extrusion
    \item Flow Contraction
\end{itemize}

\subsection{Condition Indicators Extraction}

\begin{figure}[h!]
    \centering
    \includegraphics[width=1\textwidth]{dfd_app.png}
    \caption{Diagnostic Features Designer App Interface}
    \label{fig:dfd_app}
\end{figure}

One of the reasons to use Matlab Data Ensemble format to manage the data
instead of others is to use the Diagnostic Feature Designer App
\ref{fig:dfd_app}.
This app provides an intuitive environment for extracting both statistical
condition indicators and power spectral density calculations with the
following extraction of frequency condition indicators. It is also possible
to generate Matlab functions to deploy the algorithms on a bigger scale.

\paragraph{Statistical Condition Indicators}

For every flow signal in the dataset, statistical condition indicators were calculated: 
\begin{itemize}
    \item Mean
    \item Standard deviation
    \item RMS
    \item Peak value
    \item Kurtosis
    \item Clearance factor
    \item Crest factor
    \item Impulse factor
    \item etc.
\end{itemize}


\paragraph{Frequency Domain Condition Indicators}

\begin{figure}[h!]
    \centering
    \includegraphics[width=1\textwidth]{sb_flow_spec.png}
    \caption{Welch's Power Spectral Density of the Flow Signal}
    \label{fig:flow_sp}
\end{figure}


Using Welch's power spectral density estimation \ref{fig:flow_sp}, frequency CI were
calculated: 

\begin{itemize}
\item First five peaks amplitude
\item Peaks frequencies
\item Spectrum band power
\end{itemize}


Extracted condition indicators were written to files with signals and
easily acceptable. After each data file contains complete information about
one measurement:
\begin{itemize}
    \item Measured signals
    \item Setting parameters (valves, dampers, load)
    \item Power spectrum calculated from measured signals
    \item Statistical and Frequency features extracted from signals
\end{itemize}

Moreover, a table was created, which contains all condition indicators
extracted, to prepare the train and test dataset for the classification
model.

\subsection{Condition Indicators Ranking}
The table of calculated condition indicators contains 25 statistical and
frequency CI. To train a classification model, it is good practice to
reduce the number of features or transform them with PCA algorithm and use
only first $n$ principal components, to remove linearly dependent condition
indicators.  According to section \ref{} Analysis of Variance (ANOVA),
specifically in our case Kruskal – Wallis one-way ANOVA algorithm was used. 


The result is a sorted table \ref{tab:sorted_ci} of condition indicators depending on
how much variance a particular condition indicator can describe in the
dataset.

\begin{table}[h]
        \centering
    \begin{tabular} {|c|c|c|} \hline
          & Features & Kruskal-Wallis \\ \hline
        1 & "FlowContraction\_ps\_spec/PeakAmp1" & 1.4815e+03 \\ \hline
        2 & "FlowContraction\_stats/CrestFactor" & 967.6028 \\ \hline
        3 & "FlowContraction\_ps\_spec/PeakAmp3" & 865.7571 \\ \hline
        4 & "FlowContraction\_stats/Mean" & 567.6620 \\ \hline
        5 & "FlowContraction\_ps\_spec/PeakAmp4" & 460.0924 \\ \hline
    \end{tabular}
    \caption{First Five Ranked Condition Indicators using ANOVA}
    \label{tab:sorted_ci}
\end{table}

Figure \ref{fig:sb_scatt_mat} shows the scatter plot of the first three condition
indicators for normal behavior and fault condition caused by the change of
throttle valve 1.
The first five condition indicators ranked by the ANOVA algorithm were used
for training the final model on all 20 labels.

\begin{figure}[h!]
    \centering
    \includegraphics[width=0.8\textwidth]{sb_scatter_matrix.png}
    \caption{Example of Scatter Plot with different CI}
    \label{fig:sb_scatt_mat}
\end{figure}


\subsection{Train Classification Model}
The main goal of the classification task is to train a model that can
predict the fault code or label signalized about pneumatic actuator
behavior by calculated condition indicators.

There are many classification models, but it is best to try different
variants and be satisfied with the best result from a practical point of
view.  The Classification Learner App from the Machine Learning Toolbox
tool can be used for experiments and iterative tuning of different
condition indicators and classification models. It is possible to try
several models, apply the PCA algorithm, interactively draw Scatter plot
and Confusion Matrix, and generate functions for practical applications.

\paragraph{Train, Test Datasets} By splitting data to train and test
datasets, we can ensure that the training model outcomes are valid. The
cross-validation resampling procedure to prevent model overfitting was used
during the model fitting.

\paragraph{Classification Model Performance}


Trained classification models show excellent results on the test dataset
for all three situations: using all CI, after applying the PCA algorithm
and using the first five CIs recommended by the ANOVA algorithm.  The
accuracy evaluations of the models are shown in Table
\ref{tab:classification_perfomance}.

\begin{table}[h!]
    \centering
    \begin{tabular}{|c|c|c|}
        \hline
        \textbf{approach} & \textbf{model}     &  \textbf{accuracy [\%]} \\
        \hline
        all features      &  Bagged Trees      &  99.45  \\
        PCA               &  Bagged Trees      &  95.18  \\
        ANOVA             &  Fine KNN          &  97.52  \\
        \hline
    \end{tabular}
    \caption{All Features vs PCA vs ANOVA perfomance}
    \label{tab:classification_perfomance}
\end{table}

Figure \ref{fig:conf_matrix} shows the confusion matrix from the Fine KNN
classification model by training on data using the ANOVA algorithm.  From
the confusion matrix, it is clear that combined faults in the dataset were
not observed much. However, the model can successfully resolve these fault
conditions too.

\begin{figure}[h!]
    \centering
    \includegraphics[width=1\textwidth]{sb_confusion_matrix.png}
    \caption{Fine KNN trained on ANOVA Dataset Confusion Matrix}
    \label{fig:conf_matrix}
\end{figure}

From a practical point of view, in this particular case, the use of the
ANOVA algorithm allows not only to reduce the number of CIs for prediction
on the model but also to calculate from the signal, not 25 CIs but only 5. 

Considering this fact, deploy this algorithm on a bigger scale on many
devices, where the calculation complexity plays a role, using the ANOVA
algorithm is justified.

\section{Summary All Sensors Comparison}
\todo[inline]{Rewrite text}

\subsection{Temperature}
Good results on data. But only because Ambient temperature was changed
between measurements. In one day it was warm, another colder :)
\todo[inline]{Text dependence on Ambient temperature}

\begin{figure}[h!]
    \centering
    \includegraphics[width=1\textwidth]{sb_temp.png}
    \caption{Caption}
    \label{fig:}
\end{figure}

\subsection{Encoder}
Good results, useful in simulations and compare results with Digital Twin.
Can be used in Model-Based CI. 
Digital twin can generate fault data, that will be applicable with encoder
sensor.

\subsection{Microphones}
Cheap, good results, but maybe problems with real life integration (noise
from another machines). Another problem cannot be modeled in simulation
system. For predictive purposes require data from real model.


\subsection{Accelerometers}
Not good, not bad. Can be used for classification task. But encoder has
more accuracy information.

\subsection{Proximity Sensors}
Cheap. Very correlated features.Can not be used for classification. But suitable
to detect binary classification (Health, Failed).
Only statistical features, no Frequency domain.

\subsection{Flow Sensors}
Very expensive sensors. Not so good results.

\subsection{Air Pressure}
This sensor always used, to control pressure valve. But not good results.
Maybe in combination with another sensor.

\subsection{Strain Gauge}
Expensive, Normal results of classification. But not suitable for
Simulation Model.


\subsection{Conclusion}

\begin{figure}[h!]
    \centering
    \includegraphics[width=1\textwidth]{sensors_final_bar.png}
    \caption{Comparison of sensors from accuracy/cost perspective}
    \label{fig:sensors_final_bar}
\end{figure}

\begin{table}[h]
    \centering
    \begin{tabular}{|c|c|c|c|c|c|c|c|}
        \hline
        \textbf{Sensor}   & Acc & Encoder & Flow & Mics & Pressure & Proximity & Strain \\
        \hline
        \textbf{Accuracy [\%]} & 91.6 & 96.1 & 97.2 & 95.8 & 76.6 & 80.5 & 95.0 \\
        \hline
        \textbf{Cost [czk]} & 2x 3500 & 25000 & 6000 & 3x 500 & 1000 & 2x 1000 & 15000 \\
        \hline
    \end{tabular}
    \caption{Comparison of sensors from accuracy/cost perspective}
    \label{tab:sensors_final}
\end{table}




\todo[inline]{Conclusion text, good performance}

