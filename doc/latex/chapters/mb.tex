% Artyom Voronin
%           _                     
% _ __ ___ | |__  _ __  _ __ ___  
%| '_ ` _ \| '_ \| '_ \| '_ ` _ \ 
%| | | | | | |_) | |_) | | | | | |
%|_| |_| |_|_.__/| .__/|_| |_| |_|
%                |_|              
%
% Brno, 2021


\chapter{PdM using a Simulation Model (10-15 pages)}

\section{Differences between Model-Based PdM and PdM using Digital Twin}
There is a difference between using Model-Based PdM and using Simulation
Model as a Digital Twin.

\section{Using Digital Twin to Generate Fault Data}
We can use Digital Twin to model situations that were not captured in the
original dataset or if it is hard to model some cases with real-world
hardware. As an example, we can model sensors fault such as sensor drift or
complete signal loss.

\section{Model-Based Condition Indicators}
Model-Based approach is suitable when it's difficult to identify condition
indicators using only signals. In some cases it's useful to fit some model
from data and extract condition indicators as some system parameter.

\subsection{Static and Dynamic Models}
If the system behavior can be fit from the data as a static model, than we
can extract condition variables from this model. For example, if model
was fitting to a polynomial model, than polynomial coefficients can be use
as condition indicators.

Signals showing dynamic behavior can be fitted to dynamic models such as
State-Space or AR, ARX, NLARX (Nonlinear auto recursive model) and so on.
Then condition indicators can be extracted as poles, zeros damping
coefficients from estimated model.


\subsection{Using Hammerstain-Weiner Model}
Demo using Hammerstain-Wiener Model. Fit model to position signal and
extract coefficients from model as Condition indicators. Classification.

\section{Using Simulation Model for Residuals
Estimation}\label{sec:residuals}

Another option is using the Simulink model with \textbf{prediction
error minimization function} to compute difference between Simulink model
and measured data. From this difference we can separate fault condition and
healthy operation. 


\subsection{Comparison with Nominal System Model}
\todo[inline]{Same thing as section \ref{sec:residuals}}

Compare actual system behavior with system model. This will generate some
error $e(t) = y(t) - \hat{y}(t)$. From this error residual can be generated
in form $r(t)=\Phi(u_t,y_t, \varepsilon_t,v_t,d)$ and after some decision.

\section{Using Digital Twin to Generate Prognostic Data}
Another option is to use Digital Twin to generate a system degradation
process. We can evaluate CI from sensor signal by changing a system's
mechanical properties as friction or mass flow leakage.  Another advantage
is that we can design experiments on the model to evaluate what type of
data we require from a real-world system to develop a robust algorithm.

\section{RUL}
Demo RUL using generated from model degradation dataset.





