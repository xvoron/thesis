\chapter{Demonstration Device Overview}

The case study of this thesis is the double-acting pneumatic piston, with a
pneumatic circuit and mechanical assembly driven by a piston.  Figure
\ref{} is a schematical representation of the system. Figure \ref{} is a 3D
render of the system.


There are seven types of sensors located on the system. Table 1 describes a
sensor purpose, signal name in the datastore, and the signal unit. 

\begin{table}[h]
    \centering
    \begin{tabular}{|c|c|c|c|}
\hline
\textbf{Sensor} & \textbf{Unit} & \textbf{Description} & \textbf{Name} \\
\hline
Encoder       & m     & displacement measurement               & LeverPosition \\
Encoder       & m/s   & velocity calculated from displacement  & LeverVelocity \\
Accelerometer & g     & accelerometer on moving part           & AccelerometerMovin\_axisZ/Y \\ 
Accelerometer & g     & accelerometer on static part           & AccelerometerStatic\_axisZ/Y \\ 
Flow Sensor   & l/min & air flow extrusion to A chamber        & FlowExtrusion \\
Flow Sensor   & l/min & air flow contraction from A chamber    & FlowContraction \\
\hline
    \end{tabular}
    \caption{Sensors overview}
    \label{tab:sensors_tab}
\end{table}


The dataset measured on the system contains almost five thousand thousand
measurements in different operating conditions. Each measurement includes a
10-second recording of moving the pistol up and down. This data was given
in the format of massive files with the ".mat" extension, which was divided
into files contains only one measurement.  The divided dataset is easier to
maintain, and Matlab recommends this type of datastores called Data
Ensemble \ref{}.

The measured examples are shown in figures 2,3, and 4.
