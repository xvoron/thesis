\chapter{Introduction}
Since the beginning of the industrial revolution, the complexity of
production machines and serial lines has gradually increased and requires
constant monitoring of the conditions of the systems for economic reasons.
On the other hand, critical systems such as aircraft, spacecraft,
automotive systems, nuclear reactors, and others require immediate alarm on
fault, localize occurred fault, and even more predict possible future
faults.  These requirements have become prerequisites for Fault Detection
and Analysis and Predictive Maintenance fields.

The production process always included elements of fault control and online
monitoring. From the first methods of fault detection, such as visual
inspection, today's factories move to automated systems consisting of
sensors and computing units to evaluate the faults. Sometimes it is
critical to monitor processing equipment in real-time to prevent damage
caused by fault or anomaly. Every single fault can cause a slowing down of
the production process and thus reducing the profit \cite{state_of_art_pdm}.

Device real-time monitoring algorithms have formed the Fault Detection and
Analysis (FDA) field.  FDA methods, in most cases, do not require machine
learning techniques and can detect failures, using fundamental algorithms
from Fourier analysis and trend checking algorithms to more complex
techniques such as Gaussian Mixture Models \cite{isermann_fdi}.

Due to the amount of data collected in recent years and the expansion of
data storage technology as cloud services and computation efficiency, it
has become possible to use more advanced algorithms for fault detection and
analysis. Using classification machine learning techniques, it is possible
to isolate where does the fault occur.  Another option that becomes
available with a large amount of data is to estimate the remaining useful
life (RUL) of the entire system. These techniques have led to predictive
maintenance as an effort for optimal maintenance solutions. The current
technical condition of the equipment is always available by information
extracted from measured signals. It is possible to use current system
conditions to estimate remaining useful life in time or distance
measurements such as days, kilometers, or cycles. Estimated residual
lifetime gives an option to plan maintenance concerning actual system
conditions \cite{pdm_dyn_systems}.

These remaining useful life estimation algorithms, the fault detection
methods and system modeling and identification techniques form a new
predictive maintenance field.

System modeling allows providing experiments and developing solutions
offline before physical hardware implementations. Unavailable or
challenging to implement measurements can be replaced by generated data
from the simulation model and finally helps to deploy a robust algorithm.

This thesis provides a brief introduction to fault detection and predictive
maintenance methodologies and a basic terminology. 
The \ref{ch:teor_surv} chapter describes the main goal and problems
in these areas and focuses on similarities and differences between these
two approaches.

Developing the simulation model of the double-acting pneumatic actuator and
comparing it with the real-life equipment using different approaches is
described in chapter \ref{ch:overview}, \ref{ch:fpm}, \ref{ch:alt_model}
and \ref{ch:compare}. 

The following chapter \ref{ch:sb} illustrates signal-based predictive
maintenance methods using different sensors available in a demonstration
device.  Appling preprocessing, feature extraction, and classification
model, sensors were evaluated in terms of functionality, accuracy, and
price. 

The model-based predictive maintenance techniques and simulation model
exploitation are demonstrated in chapter \ref{ch:mb}. The simulation model
is used to determine the residual signals between the measured data and the
simulation model's output. Also, using a simulation model, degradation data
are generated and used in the remaining useful life estimation.

