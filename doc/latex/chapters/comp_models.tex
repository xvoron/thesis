% -------------------------------------------------------
%  ___ ___  _ __ ___  _ __   __ _ _ __ ___ 
% / __/ _ \| '_ ` _ \| '_ \ / _` | '__/ _ \
%| (_| (_) | | | | | | |_) | (_| | | |  __/
% \___\___/|_| |_| |_| .__/ \__,_|_|  \___|
%                    |_|                   
% -------------------------------------------------------
\chapter{Models Comparison}\label{ch:compare}

As mentioned earlier \ref{sec:digital_twin}, the simulation model can be used in several
situations.  Models of the normal condition can simulate system output to a
given input in normal operating conditions. This type of model can be used
to provide, for example, residual estimation. Compare normal condition
model with measured signals from sensors decision algorithm can evaluate
possible faults. 

Suppose the model can simulate the system in different conditions. In that
case, it gives an option to implement  "What-If" simulations and prevent
fault situations that are not captured in the measured dataset.

No best solution would apply in all situations, but for a specific example
of the double-acting pneumatic actuator with the measured dataset, the more
efficient model can be evaluated. Table \ref{tab:models_compare} represents the comparison
simulation models in 4 categories, simulation speed, accuracy concerning
the actual model, the difficulty of deploying the model, the behavior under
normal conditions and the possibility of simulating abnormal "What-If"
situations.

The speed of the simulation or calculation complexity performs a more
prominent role in the model's design, especially during the estimation of
the parameters, where the simulations are performed hundreds of times in a
row.

\begin{table}[h]
    \centering
    \begin{tabular}{|c|c|c|c|c|}
\hline
\textbf{model} &\textbf{speed} &\textbf{accuracy} &\textbf{normal cond.} &\textbf{abnormal} \\
\hline
FPM            & fast          & normal           & yes                  & yes \\
Simscape       & low           & normal           & yes                  & yes \\
HW model       & fast          & very low         & -                    & - \\
NARX           & fast          & high             & yes                  & - \\
\hline
    \end{tabular}
    \caption{Models developed by different approach comparison}
    \label{tab:models_compare}
\end{table}
    
Due to the above facts, further work was continued with the help of the
first principles model, and the development of the other models was
suspended. The first principle simulation model will be used in the next chapter
\ref{ch:mb}, PdM using Simulation Model. 
All models can be found in the attachment \textit{models}; using scripts
\textit{first\_principle\_model\_perfomance.mlx,
data\_driven\_model\_perfomance.mlx}, models can
be explored interactively.

