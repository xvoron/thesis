%% Artyom Voronin
%%  __     _ _                   
%% / _| __| (_)  _ __  _ __ ___  
%%| |_ / _` | | | '_ \| '_ ` _ \ 
%%|  _| (_| | | | |_) | | | | | |
%%|_|  \__,_|_| | .__/|_| |_| |_|
%%              |_|              
%% Brno, 2021

\chapter{Theoretical Survey}\label{ch:teor_surv}


This chapter contains a short introduction to the main goals and problems
presented in fault detection and analysis and predictive maintenance
techniques. A brief review of methodologies used in these fields and
general approaches. Section \ref{sec:digital_twin} digital twin describes scenarios where a
simulation model is used in predictive maintenance and helps develop
robust, efficient algorithms.

% -----------------------------------------------------------------------------
% 
% -----------------------------------------------------------------------------

\section{Problem Definition}

% XXX FaultDetectionMethods-ALiteratureSurvay.pdf

In practice many types of machinery require some calibration and monitoring
for adequate working. An anomaly or fault detection in time can prevent
machinery from damage that causes loss of money due to non-working or
destroyed equipment.  Predicting where the fault appears reduces the cost
of diagnosis and replacement operations. The possibility of estimating the
remaining useful life allows to optimize a maintenance process and reduce
maintenance costs \cite{lit_survey}.


Smart manufacturing, the combination of sensors, the possibility of
preprocessing and extracting useful information from measurements and
decision algorithms based on this information, allows increasing production
efficiency and significantly reducing maintenance operations.


\paragraph{Types of Maintenance} There are three main types of maintenances
(fig. \ref{fig:maintenance}). Each following type of maintenance requires
increasing complexity of monitoring
and decision algorithms \cite{matlab_intro}:

\begin{itemize}
    \item Reactive maintenance, where maintenance coming after the life of
        the system is excess.
    \item Preventive maintenance is driven item by
        schedules that may keep the system safe but not optimal from an
        efficiency/cost perspective. 
    \item Predictive maintenance is an
        effort to optimize a maintenance strategy.
\end{itemize}

\begin{figure}[h!]
    \centering
    \includegraphics[width=1.1\textwidth]{maintenance.png}
    \caption{Reactive, preventive and predictive types of maintenance
    \cite{matlab_intro}}
    \label{fig:maintenance}
\end{figure}


\paragraph{Fault Types} A fault is not an acceptable deviation of at least one
characteristic or parameter of the system from the standard condition.
There are different faults by their sources. 
\begin{itemize}
    \item Plant faults appear in system
        behavior and cause manufacturing performance.
    \item Component fault
    \item Sensor faults occurred in the sensor during measurements.
    \item Combination of faults
\end{itemize}
In many cases, faults lead to a system failure and
the system is no longer able to perform required functions.
There may also be a malfunction after which the system returns to normal
operation. 

Faults can be classified by the location where they appear, by a fault
form, or based on the form in which the fault is added to the system
\cite{lit_survey}.


% -----------------------------------------------------------------------------
% 
% -----------------------------------------------------------------------------

\section{Fault Detection and Analysis (FDA)}\label{sec:fda}

Fault Detection and Analysis, FDA (Fault Detection and Isolation, FDI) is a
subfield of control engineering focused on detecting the fault and
identifying where this fault is located \cite{fdi_cern2}. 
The main goals of FDI are
\begin{itemize}
    \item Fault detection, detect anomalies in real-time
    \item Fault isolation, find the root cause
    \item Fault identification, estimation of the magnitude, type, or nature of
        the fault 
\end{itemize}

Several methods are partly overlapped but divided into two main
categories. 

\paragraph{Signal-Based methods} Signal-Based methods (SB), explore measured
data and extract useful information in the form of features
\ref{fig:signal_based}. The following methods belong to the SB approach: 

\begin{itemize}
    \item Limit and trend checking
    \item Spectral analysis
    \item Data analysis (PCA)
    \item Pattern recognition
\end{itemize}

\begin{figure}[h!]
    \centering
    \includegraphics[width=0.7\textwidth]{signal_based.pdf}
    \caption{Signal-Based Method}
    \label{fig:signal_based}
\end{figure}


\paragraph{Model-Based methods} Model-Based methods exploit models identified
from real-life systems \ref{fig:model_based}. The model-based approach is
suitable when it is difficult to gain useful information using only
measured signals. If the system structure is known, it is possible to
extract features such as state variables or some system parameters.
Another option is to compare real system behavior with nominal healthy
model and use residuals as inputs to decision algorithms \cite{matlab_full}.
Typical model-based techniques include

\begin{itemize}
    \item Residual estimation (compare measurements with "healthy" model)
    \item Polynomial coefficients
    \item State variables estimated using state observers
    \item Parameter estimation
\end{itemize}

\begin{figure}[h!]
    \centering
    \includegraphics[width=0.5\textwidth]{model_based.pdf}
    \caption{Model-Based Method}
    \label{fig:model_based}
\end{figure}


Automated fault detection depends on input from sensors and postprocessing
algorithms. In many manufacturing applications, sensor failures are the
most common equipment failure.


The result of FDI is the detection and identification of faults that occur
during the operation of the device. Subsequently, predicted faults are
processed using fault tolerance and predictive maintenance algorithms.

\textbf{Fault Tolerance}: Provide the system with the hardware architecture
and software mechanisms that will allow, if possible, to achieve a given
objective in normal operation and given fault situations \cite{fdi_cern2}.

% -----------------------------------------------------------------------------
% 
% -----------------------------------------------------------------------------


\section{Predictive Maintenance (PdM)}
\textbf{Predictive maintenance (PdM)} is cost-effective maintenance strategy that
predicts time to failure and warns of an anticipated location where this
could occur.

\subsection{Goals}
There are two main goals of predictive maintenance,  remaining useful life
(RUL) estimation and identification where the future failure can appear or
what is the reason for decreasing RUL.  As a result of PdM is RUL
representing the number of cycles, days or time before the fault occurred.
And the probability of when or where this fault can appear
\cite{matlab_full}. 

\subsection{Overview of the PdM development workflow}

\begin{figure}[h!]
    \centering
    \includegraphics[width=1\textwidth]{pdm_deploy_sequence.pdf}
    \caption{Predictive maintenance development sequence}
    \label{fig:pdm_dev_seq}
\end{figure}


Figure \ref{fig:pdm_dev_seq} represents the recommended PdM development
workflow.  The development of predictive maintenance algorithms starts with
raw measured signals from sensors. For further working with data, it is a
good manner to combine measurements to a dataset with a logical structure
of elements. In this thesis, a common data ensemble structure was used.
Each measurement has its own data file with all measured signals at a
particular time. 


If collected data require some preprocessing techniques as data cleaning,
smoothing or filter the signal, detrend, normalizing, etc., it can be done
at this step. 

The next step is to extract condition indicators using predictive
maintenance methods described in \ref{sec:ci}. As long as the optimal
solution is not found, try to figure out the best combination of condition
indicators described in \ref{sec:ci_ranking} and train different
classification models iteratively.  After the efficient solution is found,
deploy the algorithm to work recursively with the study-case system.


\subsection{Condition Indicators}\label{sec:ci}
In the prediction maintenance field, features extracted from measured
signals are called \textbf{Condition Indicators, CI}. 

Condition Indicators
represent some system behavior and hide information about system operation
conditions.  Generally, CI is represented by three main domains. There is a
time domain, frequency domain, time-frequency domain. But in fact, CI can
be any system parameter or value corresponding with the system's current
condition \cite{matlab_full}.

The methods of extracting condition indicators from the signal are defined
in the same way as in FDI \ref{sec:fda}. 

The \textbf{signal-based approach} is suitable when we have measurements
from the system in different operating conditions.  But there is a problem
that signal-based approach enables classifying and learning just the
patterns observed in the training dataset.  On the other hand, the
\textbf{model-based approach} uses physical failure models and does not
require a large dataset of failure data. And they may work in situations
never observed in data before. Moreover, the model-based method is helpful
in case the measured signal has a more complex relationship with the input
signal.

Between common signal-based CI belongs:
\begin{itemize}
    \item Time-domain: mean, standard deviation, RMS, skewness, etc. 
    \item Frequency-domain: mean frequency, peak values/frequencies, power
        bandwidth, etc. 
    \item Time-frequency domain: Spectral entropy/kurtosis, moments, etc. 
\end{itemize}

Model-based approach use model properties such as:
\begin{itemize}
    \item poles and zeros location
    \item damping coefficient
    \item state variables values
    \item modal analysis
    \item residual values
\end{itemize}

\subsection{Condition Indicators Ranking}\label{sec:ci_ranking}
Multiple condition indicators can be extracted from each sensor signal.  A
good practice to reduce the number of CI and keep only those which provide
essential information. 

One of the possibilities is applying Principle Component Analysis (PCA) to
transform features from one coordinate system to a new orthogonal basis.
Data reduced by using the first n principal components that optimally
describe the variance of the dataset. Applying the PCA algorithm still
requires the extraction of all condition indicators from the signal.

Another option is to rank the futures using the Analysis of Variance
(ANOVA) algorithm. This algorithm describes relations among CI in the form
of their mean values. The result gives information about how much
particular CI represents data. Using the first n CI, we reduce the number
of CI and reduce the number of extracted features from measured signals.
This fact means that using ANOVA reduced the time and complexity of
calculations \cite{matlab_full}.

\subsection{Fault Classification}
Classification models are used to recognize faults from a set of CI. The
set of CI must contain labels that determine the current condition of the
device in the form of fault code, string, etc. The correlation between
different CI can be explored using a 2D or 3D scatter plot. The model
performance is usually represented by total accuracy and confusion matrix,
where on one axis there are true labels and on the other there are
predicted from the model. The common types of classification models are:

\begin{itemize}
    \item Decision Trees
    \item Supported Vector Machines (SVM)
    \item Neigherest Neighbors (KNN)
    \item Ensemble Classifiers
    \item Neural Networks (ANN)
\end{itemize}

A good practice is to divide an original dataset of CI into train,
validation and test sub-datasets to prevent model overfitting. Choosing the
best classification model depends on training data and requires experiments
with different models.

\subsection{Remaining useful life}

The remaining useful life (RUL) is the expected time remaining before the
machine requires repairment or replacement, and it is a central goal of
PdM.

The problem of estimating the remaining useful life is connected with
evaluating condition indicators associated with the system's degradation
process. These condition indicators must satisfy the requirements for
monotonicity, trendability, and prognosability \cite{matlab_full}.

The models used to estimate the remaining useful life depend on the
historical data which are available. There are three types of possible
models. 

\paragraph{Survival model}
The survival model is considered when we have only failure data available,
but the whole degradation history is not recorded. The probability density
function can be obtained from failure data and used to estimate RUL.

\paragraph{Degradation model}
The degradation model gives an option to estimate RUL based on data without
failure moment captured but only recorded degradation process. In this
situation, it is necessary to determine a safety threshold that CI must not
cross.

\paragraph{Similarity model}
In case we have the whole history of the degradation process of similar
systems, including failure, the similarity model can be used. The upcoming
CI is compared with historical degradation paths obtained from the training
dataset and the best similarity trend is evaluated as RUL value.

\section{Digital Twin}\label{sec:digital_twin}
A digital twin is a digital representation of the real-life system. It can
be represented as a component, a system of components, or as a system of
systems. 

A digital twin can be updated with incoming data from sensors.
Fitting the model to new data, the digital twin represents the current
condition state of the real-world object.  There are many advantages of
using models in PdM. A digital twin can hold historical data about system
behavior. Apart from this, it can be used for simulation system operation
in different conditions, designing control and simulating future behavior
(RUL, "What-if"). The dataset extended by data from the simulation model
represents synthetic dataset. This dataset type can contain different
measured fault and healthy data of the system and hard to realizable in
real-world fault situations \cite{digit_twin}. 

A mathematical model of the real-world system can be created using
different approaches. 
\begin{itemize}
    \item First-principles modeling requires an understanding of the
        fundamental process of the system.
    \item Physical modeling (Simscape).
    \item Data-driven modeling where the system is represented as a Blackbox.
    \item Combination of multiply approaches.
\end{itemize}

\section{Comparison PdM and FDA approaches}

\begin{figure}[h!]
    \centering
    \includegraphics[scale=0.3]{FDI_PM.png}
    \caption{Relative arrangement of PdM and FDI algorithm \cite{fdi_cern2}}
    \label{fig:fdi_pm}
\end{figure}


Figure \ref{fig:fdi_pm} presents a relative arrangement of Predictive Maintenance
(PdM) and Fault Detection and Identification (FDI or FDA) algorithms. From
the figure, it is clear that Predictive Maintenance is an extension of the
FDI approach, with recommended workflow techniques suitable for optimizing
system maintenance.


Both methods are closely overlapped and use quite similar techniques.
However, predictive maintenance over the FDA is extended by RUL estimation.
And it leads not only to fault detection and monitoring at a given moment
but also to the possible prediction of a fault in the near future. 

\section{Applications}\label{sec:applications}
% https://www.reliableplant.com/Read/12495/preventive-predictive-maintenance

The most significant interest in PdM is the manufacturing sector that
requires efficiency maintenance strategies to increase productivity and
reduce money-lost \cite{state_of_art_pdm}. The PdM is used in the field
that is highly dependent on safety types of machinery such as aircraft or
rail industry. Using the PdM condition monitoring, it is possible to
prevent unexpected fails. The oil and gas industry supports the PdM field;
due to the amount of data collected in these industries, the PdM techniques
are beneficial.
