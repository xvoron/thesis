\chapter{Conclusion}

\section{Simulation Model}
One of the outcomes from the thesis is a simulated model built based on
differential equations from the pneumatic-mechanical domain, modeled and
developed using Matlab/Simulink software. The simulation model was
estimated with parameters of healthy system behavior. However, there is an
option to reestimate parameters to fault state and simulate the system in a
fault condition. 

Due to the available measured data and significantly nonlinear dynamics of
the system, the simulation model shows good agreement with the measured
data. In contrast to the model built using SimScape library, it is
distinctly less computationally expensive while maintaining numerical
stability. These facts are fundamental when parameter estimation is in
progress.

The simulation model can be used to experiment with the system's behavior
in different conditions, model fault situations and generate data for the
design and development of the robust predictive maintenance algorithms. 


\section{Signal-Based PdM}
Another outcome is verifying the possibility of classification and
detection of a fault condition applying predictive maintenance techniques,
using signal-based and model-based methods.

The experiments were performed on a dataset measured on a demonstration
device using seven types of sensors.
  

A signal-based method is based on the extraction of useful information
directly from the signal in time-frequency domains. Each sensor required an
individual approach for preprocessing, extracting features, ranking
features, and building the classification models. But generally, there is
minimal preprocess needed to keep the possible helpful information. 

The table \ref{tab:sensors_final} contains the comparison of sensors in 2 categories, accuracy
performed in the test dataset and sensor cost. The graph
\ref{fig:sensors_final_bar} visualizes these data.

Surprisingly, all sensors showed an accuracy of more than 75\%. Microphones
offer excellent performance from a cost/accuracy perspective, and they are
suitable for installation and maintenance.

\begin{figure}[h!]
    \centering
    \includegraphics[width=1\textwidth]{sensors_final_bar.png}
    \caption{Comparison of sensors from accuracy/cost perspective}
    \label{fig:sensors_final_bar}
\end{figure}

\begin{table}[h]
    \centering
    \begin{tabular}{|c|c|c|c|c|c|c|c|}
        \hline
        \textbf{Sensor}   & Acc & Encoder & Flow & Mics & Pressure & Proximity & Strain \\
        \hline
        \textbf{Accuracy [\%]} & 91.6 & 96.1 & 97.2 & 95.8 & 76.6 & 80.5 & 95.0 \\
        \hline
        \textbf{Cost [czk]} & 2x 3500 & 25000 & 6000 & 3x 500 & 1000 & 2x 1000 & 15000 \\
        \hline
    \end{tabular}
    \caption{Comparison of sensors from accuracy/cost perspective}
    \label{tab:sensors_final}
\end{table}

\section{Model-Based PdM}
The next part of this thesis was to apply model-based methods and using a
simulation model for predictive maintenance algorithms. These algorithms
are practical when it's hard to extract useful information using a
signal-based method. Or it's suitable in some cases where we understand
the system dynamics and know how to exploit some system variables as
condition indicators.

The use of the method of extraction features in the form of a Nonlinear
system identification model coefficient, specifically with the
Hammerstein-Wiener model, did not give reliable results. Extracted features
have no statistical addiction, and it's impossible to predict fault type
using this method on the measured data from the pneumatic piston as a case
study.

On the other hand, the residual estimation using the simulation model
showed excellent results. The measured position signal was compared with
the signal from the simulation model. This residual signal was used to
classify the fault condition and achieve  99 \% on a smaller dataset.  But
given the results obtained using the signal-based method, the residual
estimation method may seem unnecessary. In this particular case, from a
practical point of view, the improvement of the result by a few percent
does not bring fundamental changes, but the calculation time increases
significantly. 

The possibility of modeling and simulation sensor faults was also verified
using the simulation model. Although it is difficult to collect data from
the sensor fault in real-life conditions, fault data can be generated from
the simulation model and even combined with the primary dataset to create a
synthetic dataset.

\subsection{RUL}
One of the main goals of predictive maintenance is to estimate the
remaining useful life. The original dataset does not contain a record of
historical data that shows degradation behavior. 

A common problem in the maintenance of pneumatic actuators is the leakage
of air from the chamber by the piston. This situation was modeled on the
simulation model and generated data used for RUL estimation. 

In the demonstration example, a flow signal was measured. From the
measurements, the shape factor parameter was calculated and used as a
condition indicator. The generated dataset contains 25 simulations with
different failure dynamics. Each measurement includes various 10 seconds
cycles, depending on the failure dynamic, before the system failure occurs.
The outcome is that it is possible to estimate the remaining useful life on
generated degradation dataset by using the residual similarity model,
pairwise similarity model, and linear degradation model. The prediction
results are satisfying.

\section{Further Development}
As a further development, it would be appropriate to estimate the modeled
system parameters piecewise to improve the results, emphasizing the
characteristics of throttle valves and dampers with adjustments. 

Perform air leak fault condition measurements and collect historical
degradation data from a real pneumatic piston. Subsequently, evaluate the
dynamics of the failure caused by the air leak. Verify the possibility of
estimating the remaining useful life using a flow sensor. An interesting
case study could be to verify if it is possible to estimate RUL using
microphones.

If the performance of the available sensors will be deficient, the pressure
measurements in the chamber can be performed. An example from the
simulation model is in section TODO.



