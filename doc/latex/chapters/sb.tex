% Artyom Voronin
%     _                     
% ___| |__  _ __  _ __ ___  
%/ __| '_ \| '_ \| '_ ` _ \ 
%\__ \ |_) | |_) | | | | | |
%|___/_.__/| .__/|_| |_| |_|
%          |_|              
%
% Brno, 2021


% ----------------------------------------------------------------------------- 

% ----------------------------------------------------------------------------- 

\chapter{Signal-Based PdM}
Signal-based predictive maintenance.
\section{Data Management and Preprocessing}
Before the final solution was developed in the whole dataset, the smaller
the reduced dataset was used for experiments and planning algorithms. 

\subsection{Data storage}
\paragraph{Manage Data}
First, a folder structure was created to collect all measured and
calculated data. The measured signals were given in 6 large files with a
".mat" extension and divided into smaller files with only one measurement
each.  The divided datastore is easier to maintain and Matlab recommends
this type of datastores called Data Ensemble \ref{}.  The full dataset
contains 4840 measurements. Each measurement includes a 10-second recording
of all signals collected from moving the piston up and down.
\paragraph{Labels}
The whole dataset was divided into 20 Labels by place of fault accumulate: 
\begin{itemize}
    \item Healthy
    \item Throttle valve 1
    \item Throttle valve 2 
    \item Small damper bottom
    \item Small damper upper
    \item Large dampers 
    \item And combinations of these faults
\end{itemize}

\subsection{Data Exploration}
Data from each sensor were explored in an attempt to find measurement
errors or anomalies in data.  Figure \ref{} shown an example of the
flow signal in different operation conditions. 

There are eight types of sensors:
\begin{enumerate}
    \item Linear encoder
    \item Flow sensor
    \item Pressure sensor
    \item Temperature sensor
    \item Accelerometer
    \item Strain gauge
    \item Microphones
    \item Proximity sensors
\end{enumerate}

\subsection{Preprocessing}
Some signals, such as Encoder, are very accurate. There is no preprocessing
needed to apply. Signals noisier (pressure signal or strain) has to be
preprocessed and applied some noise reduction algorithms. However, during
experiments turned out that non preprocessed signals have better
performance. For example, the preprocessed pressure classification model
gives 78 \% accuracy; the raw pressure signal gives approximately 82 \%.

\section{SB methods and Flow Sensor as an Example}
In this section, signal-based methods were applying to the flow sensor as a
case study example.  The rest of the sensors was processed in the same way;
however, each required an individual approach.

\subsection{Flow Sensor Data}
There are two flow signals in the datastore. Both are connected to port A
in scheme \ref{}.
\begin{itemize}
    \item Flow Extrusion
    \item Flow Contraction
\end{itemize}

\subsection{Condition Indicators Extraction}
\paragraph{Statistical Condition Indicators}
From every signal in the dataset, statistical features were calculated.

\paragraph{Frequency Domain Condition Indicators}
Using Welch's power spectral density estimation, frequency features were
calculated. 


Extracted condition indicators were written to files with signals and
easily acceptable. After each data file contains complete information about
one measurement:
\begin{itemize}
    \item Measured signals
    \item Setting parameters (valves, dampers, load)
    \item Power spectrum calculated from measured signals
    \item Statistical and Frequency features extracted from signals
\end{itemize}

Moreover, a table was created, which contains all condition indicators
extracted, to prepare the train and test dataset for the classification
model.

\subsection{Features Ranking}
Kruskal-Wallis ANOVA

\subsection{Train Classification Model}

\paragraph{Split CI to train and test}

\paragraph{Classification Model Performance}

\section{Summary All Sensors Comparison}




%Dataset was divided to 5 main categories.
%
%
%Data has been accumulated to ".mat" files.
%Each file contains signals from sensors during 10 seconds measurements with
%different pneumatic actuator configuration. Example results from one
%experiment are represented in figures \ref{fig:data_exmp1},
%\ref{fig:data_exmp2}. 
%
%
%\section{Data management}
%Before the final solution was developed in the whole dataset, the smaller
%reduced dataset was used for experiments and planning algorithms. 
%
%\paragraph{Data Ensembles}
%Data files have been reshaped to Data Ensembles format used for Condition
%monitoring purposes. This format allows processing data without copying the
%whole dataset to memory at once but processes them one by one. In large datasets
%it gives an option to manipulate with data without problems with allocated memory.
%
%Divided to 3 datasets:
%\begin{itemize}
%    \item Train data
%    \item Validation data
%    \item Test data
%\end{itemize}
%
%\section{Preprocessing}
%Measured signals require preprocessing concerning the preservation of the information
%base. For smoothing data Moving Average function were used.
%As an example, the figure \ref{fig:preprocess} is shown the "raw" and filtered signals.
%The whole dataset of preprocessed data is relatively big. For
%time-saving, parallel computing was used for all computationally
%demanding parts of the code.
%
%
%
%\section{FDI methods}
%
%\subsection{Line checking}
%
%We can use Proximity sensor time delay between input signal and upper
%proximity sensor signal to evaluate if there is some fault.
%
%Same with Position, if not reach some end position, there is a fault.
%
%Flow sensor, check if the float mean value is under some threshold, there
%is fault.
%
%
%\section{Condition Indicators extraction}
%
%For classification task purpose from the signals have been extracted
%statistical features such as mean, median, peak to peak value, etc.
%As a condition "FaultCode" variable
%were used. This variable represent configuration of pneumatic actuator
%during the measurement.
%
%All calculated features were added to the dataset and were ranked by
%Kruskal-Wallis ANOVA algorithm. Following table \ref{tab:feat} contain
%5 first best features ranked for classification purpose.
%
%Kruskal-Wallis is very suitable to ranking features before using PCA or
%SVD.
%
%\paragraph{Selecting Condition Indicators} There is a problem if we will
%deploy classification task with large features dataset.
%There are different possibilities to reduce data before train
%classification model or do a prediction. On of them is to rank a features
%by Analysis of Variation algorithm to evaluate a good representation
%features.
%
%
%% PCA vs Sort features(Anova) 
%
%
%\subsection{Microphones}
%Cheap, good results, but maybe problems with real life integration (noise
%from another machines). Another problem cannot be modeled in simulation
%system. For predictive purposes require data from real model.
%
%\subsection{Encoder}
%Good results, useful in simulations and compare results with Digital Twin.
%Can be used in Model-Based CI. 
%Digital twin can generate fault data, that will be applicable with encoder
%sensor.
%
%\subsection{Acceleration sensors}
%Not good, not bad. Can be used for classification task. But encoder has
%more accuracy information.
%
%\subsection{Proximity Sensors}
%Cheap. Very correlated features.Can not be used for classification. But suitable
%to detect binary classification (Health, Failed).
%Only statistical features, no Frequency domain.
%
%\subsection{Flow Sensors}
%Very expensive sensors. Not so good results.
%
%\subsection{Air Pressure}
%This sensor always used, to control pressure valve. But not good results.
%Maybe in combination with another sensor.
%
%\subsection{Strain Gauge}
%Expensive, Normal results of classification. But not suitable for
%Simulation Model.
%
%\subsection{Temperature}
%Good results on data. But only because Ambient temperature was changed
%between measurements. In one day it was warm, another colder :)
%
%
%\section{Classification Task}
%
%The main goal of the classification task is to train a model that can
%predict the "FaultCode", or "Label" signalized about pneumatic actuator behavior by
%calculated features.
%
%Using Kuskal-Wallis one way analysis of variance, features were ranked by
%importance with respect to correlation. This gives opportunity to reduce
%number of features before PCA analysis.
%
%Principal component analysis (PCA) has been used to reduce the number of
%features and chose the best representants.
%
%The trained model has been exported to \textbf{models/} directory.
%
